\begin{frame}
  \frametitle{Testing Methods}
  \begin{tikzpicture}[overlay]

    \node[bkg,anchor=north west,text width=3.2cm,text depth=6cm,label=above:{\color{Base04}Static}] at (-0.5,2.3) {};
    \node[bkg,anchor=north west,text width=6.8cm,text depth=6cm,label=above:{\color{Base04}Dynamic}] at (3.8,2.3) {};

    \visible<2->{

    \node[nrm2,short,anchor=north west] at (-0.2,1.9) {Reviews};
    \node[nrm2,short,anchor=north west] at (-0.2,0.7) {Walkthroughs};
    \node[nrm2,short,anchor=north west] at (-0.2,-0.5) {Inspections};
    \node[nrm2,short,anchor=north west] at (-0.2,-1.7) {IDE (syntax)};

    }

    \visible<2-4>{
      \node[nrm2,short,anchor=north west] at (-0.2,-2.9) {Analysis};
    }

    \visible<3->{

    \node[nrm2,long,anchor=north west] (F) at (4.1,1.9) {Functional};
    \node[nrm2,long,anchor=north west] (N) at (4.1,-1.6) {Non-functional};

    }

    \coordinate (o) at (6,0.7);
    \coordinate (p) at (6,-2.8);
    \coordinate (g) at (7.52,-3);

    \visible<4>{

      \node[nrm2,tiny,anchor=north west] (a) at (4.1,0.5)  {Unit};
      \node[nrm2,tiny,anchor=north west] (b) at (5.1,-0.4) {Integration};
      \node[nrm2,tiny,anchor=north west] (c) at (6.72,0.5)  {System};
      \node[nrm2,tiny,anchor=north west] (d) at (7.8,-0.4) {Regression};

    }

    \visible<5>{

    \node[lit,tiny,anchor=north west] (a) at (4.1,0.5)  {Unit};
    \node[lit,tiny,anchor=north west] (b) at (5.1,-0.4) {Integration};
    \node[lit,tiny,anchor=north west] (c) at (6.72,0.5)  {System};
    \node[lit,tiny,anchor=north west] (d) at (7.8,-0.4) {Regression};
    \node[lit,short,anchor=north west] at (-0.2,-2.9) {Analysis};
    \node[lit,tiny,anchor=north west] (e) at (9.4,0.5)  {Sanity};

    }

    \visible<4>{
      \node[nrm2,tiny,anchor=north west] (e) at (9.4,0.5)  {Sanity};
    }

    \visible<4->{

    \node[nrm2,tiny,anchor=north west] (f) at (4.1,-3)  {Performance};
    \node[nrm2,tiny,anchor=north west]     at (6.95,-3) {Usability};
    \node[nrm2,tiny,anchor=north west] (h) at (9.18,-3)  {Security};

    \path[draw,color=Base06,-latex] (F) |- (o) -| (a);
    \path[draw,color=Base06,-latex] (F) |- (o) -| (b);
    \path[draw,color=Base06,-latex] (F) |- (o) -| (c);
    \path[draw,color=Base06,-latex] (F) |- (o) -| (d);
    \path[draw,color=Base06,-latex] (F) |- (o) -| (e);

    \path[draw,color=Base06,-latex] (N) |- (p) -| (f);
    \path[draw,color=Base06,-latex] (N) |- (p) -| (g);
    \path[draw,color=Base06,-latex] (N) |- (p) -| (h);

    }

  \end{tikzpicture}
\end{frame}

\begin{frame}
  \frametitle{The ``Box'' Approach}
  Used in tandem with testing methods.
  \vspace{0.5cm}
  \begin{itemize}
  \item Describes the point-of-view of the tester.
  \vspace{0.2cm}
  \item Black-box testing.
  \vspace{0.2cm}
  \item White-box testing.
  \vspace{0.2cm}
  \item Grey-box testing.
  \end{itemize}
\end{frame}

\begin{frame}
  \frametitle{Black-box Testing}
  We don't know anything about internals, only
  the interfaces.
  \begin{tikzpicture}[overlay]
    \node[class] (A) at (-9,-1) {\textbf{Interface}\nodepart{second}defined};
    \node[class] (B) at (-4,-1) {\textbf{Implementation}\nodepart{second}?};
  \end{tikzpicture}
\end{frame}

\begin{frame}[fragile]
  \frametitle{Black-box Testing}
  \begin{block}{Function definition}
    \begin{lstlisting}[style=C]
  // Perform Newton-Raphson solve.
  // Solution is bounded by [x1,x2].
  // Initial value is x.
  // Solve tolerance is tol, maximum iterations are
  // max_its.
  template< class FuncT, class T >
  T newton( FuncT func, T x1, T x2, T x,
            T tol, int max_its );
    \end{lstlisting}
  \end{block}
\end{frame}

\begin{frame}[fragile]
  \frametitle{Black-box Testing}
  \begin{block}{Potential tests}
    \begin{itemize}
    \item Check known functions work ($x$, $x^2$, etc).
    \item Check variation of lower/upper bounds.
    \item Check variation of initial x.
    \item Check variation of tolerance.
    \item Check variation of maximum iterations.
    \end{itemize}
  \end{block}
  \pause
  \vspace{1cm}
  {\LARGE\color{Base0A} Can we do better?}
\end{frame}

\begin{frame}
  \frametitle{White-box Testing}
  Have access to application internals.
  \begin{tikzpicture}[overlay]
    \node[class] (A) at (-5,-1) {\textbf{Interface}\nodepart{second}defined};
    \node[class] (B) at (0,-1) {\textbf{Implementation}\nodepart{second}defined};
  \end{tikzpicture}
\end{frame}

\begin{frame}[fragile]
  \frametitle{White-box Testing}
  \begin{block}{Function implementation}
    \begin{lstlisting}[style=C]
  template< class FuncT, class T >
  T newton( FuncT func, T x1, T x2, T x,
            T tol, int max_its ) {
    ...
    // Use a scaling to make sure we don't
    // converge simply because x is very small.
    T scale;
    if( x > 0.0 )
      scale = 1.0/x;
    else
      scale = 1.0;
    ...
  }
    \end{lstlisting}
  \end{block}
\end{frame}

\begin{frame}[fragile]
  \frametitle{White-box Testing}
  \begin{block}{More thorough tests}
    \begin{itemize}
    \item Check systems with small scales converge correctly.
    \item Check systems with large scales converge correctly.
    \item Check initial value of 0 works as expected.
    \end{itemize}
  \end{block}
  \pause
  \vspace{1cm}
  {\LARGE\color{Base0A} Perhaps too much effort?}
\end{frame}

\begin{frame}[fragile]
  \frametitle{Grey-box Testing}
  At the level of black-box testing with some awareness of internals.
  \begin{tikzpicture}[overlay]
    \node[class] (A) at (-9.5,-1) {\textbf{Interface}\nodepart{second}defined};
    \node[class] (B) at (-4.5,-1) {\textbf{Implementation}\nodepart{second}aware of};
  \end{tikzpicture}
\end{frame}
