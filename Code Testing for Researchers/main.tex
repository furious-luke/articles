\documentclass{beamer}

\usepackage{listings,tikz,hyperref}
\usetikzlibrary{calc,arrows,decorations.markings,positioning}
\usetikzlibrary{shapes.arrows,shapes}

% Setup epigraph.
\usepackage{epigraph,etoolbox}
\setlength\epigraphwidth{10cm}
\setlength\epigraphrule{0.5pt}
\makeatletter
\patchcmd{\epigraph}{\@epitext{#1}}{\itshape\@epitext{#1}}{}{}
\makeatother

\tikzset{
  class/.style={
    rectangle,draw=Base02,text centered,anchor=north west,text=Base02,minimum width=3cm,shading=axis,bottom color=Base03!80,top color=Base06,shading angle=45,rectangle split,rectangle split parts=2
  },
  nrm/.style={
    draw=none,
    rectangle,
    rounded corners=2mm,
    inner sep=3mm,
    fill=Base02,
    ultra thick,
    minimum width=5cm,
    text=Base04
  },
  nrm2/.style={
    draw=none,
    rectangle,
    rounded corners=2mm,
    inner sep=3mm,
    fill=Base02,
    ultra thick,
    text=Base04
  },
  short/.style={
    minimum width=3.2cm,
    text depth=0.07cm
  },
  long/.style={
    minimum width=6.8cm,
    text depth=0.07cm
  },
  tiny/.style={
    minimum width=1.5cm,
    minimum height=0.75cm,
    inner sep=2mm
  },
  lit/.style={
    draw=Base0D,
    rectangle,
    rounded corners=2mm,
    inner sep=3mm,
    fill=Base0D,
    ultra thick,
    minimum width=5cm,
    text=Base07
  },
  bad/.style={
    draw=Base08,
    rectangle,
    rounded corners=2mm,
    inner sep=3mm,
    fill=Base08,
    ultra thick,
    minimum width=5cm,
    text=Base07
  },
  exp/.style={
    draw=Base0D,
    rectangle,
    rounded corners=2mm,
    inner sep=3mm,
    fill=Base02,
    ultra thick,
    text width=4cm,
    text=Base07,
    text depth=4.7cm
  },
  bkg/.style={
    draw=Base03,
    rectangle,
    rounded corners=2mm,
    inner sep=3mm,
    fill=none,
    ultra thick,
    text=Base03,
    dashed
  },
  arr/.style={
    draw=Base0D,
    fill=Base0D,
    line width=2pt,
    dashed
    %% decoration={markings,mark=at position 1 with {\arrow[scale=2,thin,Base0D]{triangle 45}}},
    %% shorten >= 8pt,
    %% preaction={decorate},
    %% postaction={draw,line width=5pt}
  },
  bb/.style={
    rectangle,
    draw=none,
    fill=Base00,
    inner sep=3mm,
    ultra thick,
    rounded corners=1mm,
    text width=3cm,
    text=Base04,
    align=center
  },
  wb/.style={
    rectangle,
    draw=none,
    fill=Base06,
    inner sep=3mm,
    ultra thick,
    rounded corners=1mm,
    text width=3cm,
    text=Base02,
    align=center
  },
  gb/.style={
    rectangle,
    draw=none,
    fill=Base03,
    inner sep=3mm,
    ultra thick,
    rounded corners=1mm,
    text width=3cm,
    text=Base06,
    align=center
  }
}

%% \lstset{
\lstdefinestyle{C}{
  language=C++,
  %% numbers=left,
  showstringspaces=false,
  formfeed=\newpage,
  tabsize=4,
  basicstyle=\footnotesize\ttfamily,
  commentstyle=\color{Base08}\itshape,
  keywordstyle=\color{Base0D},
  stringstyle=\color{Base0B},
  morekeywords={std, thrust, include, ifndef, define, endif}
  %% morekeywords={models, lambda, forms, dict, list, str, import, dir, help,
  %%  zip, with, open}
}
%% \lstset{
\lstdefinestyle{Py}{
  language=Python,
  %% numbers=left,
  showstringspaces=false,
  formfeed=\newpage,
  tabsize=4,
  basicstyle=\footnotesize\ttfamily,
  commentstyle=\color{Base08}\itshape,
  keywordstyle=\color{Base0D},
  stringstyle=\color{Base0B},
  morekeywords={assert}
  %% morekeywords={models, lambda, forms, dict, list, str, import, dir, help,
  %%  zip, with, open}
}
%% \lstdefinestyle{C}{language=C}
%% \lstdefinestyle{Py}{language=Python}

\mode<presentation>{\usetheme{Median16}}
\title[Software Testing for Researchers]{Software Testing for Researchers}
\author{Luke Hodkinson}
\institute{\hfill Center for Astrophysics and Supercomputing \\
  \hfill Swinburne University of Technology \\
  \hfill Melbourne, Hawthorn 32000, \underline{Australia} }

\date{\today}
\titlegraphic{
  \begin{tikzpicture}[overlay]
    \node[anchor=south west,inner sep=0pt] at ($(current page.south west)+(7cm,2cm)$) {
      \includegraphics[width=3cm]{swinburne.png}
    };
  \end{tikzpicture}
  \vspace{-\baselineskip}
}

\begin{document}

\frame{\titlepage}

\begin{frame}
  \frametitle{Motivation}
  Are any of these phrases familiar?
  \begin{itemize}
    \pause
    \item It was working just last week!
    \pause
    \item I only made a small change, I don't know how it could have broken \emph{that}...
    \pause
    \item I'm sure I've fixed this before.
    \pause
  \end{itemize}
  \vspace{1cm}
  {\Large\color{Base09}\hspace{2cm}\emph{Software testing can help}}
\end{frame}

\begin{frame}
  \frametitle{Motivation}
  \begin{itemize}
    \item Testing will {\color{Base09}\emph{save}} you time.
    \pause
    \item Testing can prevent problems, not just identify them (TDD).
    \pause
    \item Testing will make your code more attractive to others.
    \pause
    \item Testing is invaluable in a team environment.
  \end{itemize}
\end{frame}

\begin{frame}
  \frametitle{Time: Thinking Long Term}
  \begin{tikzpicture}[overlay]
    \node[lit,minimum width=10.7cm,anchor=north west] at (0,-3) {Library};
    \node[lit,minimum width=10.7cm,anchor=north west] at (0,-0.5) {Application};
    \node[single arrow,draw=none,fill=Base09,rotate=-90,anchor=north west] at (2.8,-1.9) {uses};
    \node[single arrow,draw=none,fill=Base09,rotate=-90,anchor=north west] at (8.3,-1.9) {uses};
    \node[single arrow,draw=none,fill=Base09,rotate=-90,anchor=north west] at (2.9,1) {inputs};
    \node[single arrow,draw=none,fill=Base09,rotate=90,anchor=north west] at (7.8,0.1) {outputs};
  \end{tikzpicture}
  \def\eyepath{(-3,0) .. controls (-2,1.8) and (2,2.2) .. (2.7,0) .. controls (2,-1.2) and (-2,-1.4) .. (-3,0)--cycle;}
  \begin{tikzpicture}[overlay,scale=0.25,shift={(31.5,8)}]
    \clip\eyepath;
    \filldraw[color=orange!50!black] (-.2,.2) circle (1.5);
    \fill[color=black] (-.2,.2) circle (0.7);
    \fill[color=white] (.3,.5) circle (0.2);
    \draw[very thick]\eyepath;
  \end{tikzpicture}
\end{frame}

\begin{frame}
  \frametitle{Time: Thinking Long Term}
  \begin{tikzpicture}[overlay]
    \visible<1>{
      \node[lit,minimum width=10.7cm,anchor=north west] at (0,-3) {Library};
    }
    \visible<2-5>{
      \node[nrm,minimum width=10.7cm,anchor=north west] at (0,-3) {Library};
    }
    \visible<2->{
      \node[single arrow,draw=none,fill=Base09,rotate=-90,anchor=north west] at (2.8,-1.9) {uses};
    }
    \visible<2>{
      \node[lit,minimum width=5.15cm,anchor=north west] at (0,-0.5) {Application};
    }
    \visible<3->{
      \node[nrm,minimum width=5.15cm,anchor=north west] at (0,-0.5) {Application};
      \node[single arrow,draw=none,fill=Base09,rotate=-90,anchor=north west] at (8.3,-1.9) {uses};
    }
    \visible<3>{
      \node[lit,minimum width=5.15cm,anchor=north west] at (5.5,-0.5) {Library};
    }
    \visible<4->{
      \node[nrm,minimum width=5.15cm,anchor=north west] at (5.5,-0.5) {Library};
      \node[single arrow,draw=none,fill=Base09,rotate=-90,anchor=north west] at (8.3,0.6) {uses};
    }
    \visible<4>{
      \node[lit,minimum width=5.15cm,anchor=north west] at (5.5,2) {Application};
    }
    \visible<5->{
      \node[nrm,minimum width=5.15cm,anchor=north west] at (5.5,2) {Application};
      \node[nrm,fill=Base08,minimum width=10.7cm,anchor=north west] at (0,-3) {Modified Library};
    }
  \end{tikzpicture}
\end{frame}

\begin{frame}
  \frametitle{Preparation}
  \includegraphics[width=\textwidth]{supercomputer.png}
  \begin{itemize}
  \item Code needs to be ready to run ASAP.
  \item Test as much as possible before hand.
  \end{itemize}
\end{frame}

\begin{frame}
  \frametitle{Working in Teams}
  \begin{tikzpicture}[
      >=stealth,
      node distance=2.5cm,
      database/.style={
        cylinder,
        cylinder uses custom fill,
        cylinder body fill=Base09,
        cylinder end fill=Base0A,
        shape border rotate=90,
        aspect=0.25,
        draw,
        text=Base06
      },
      user/.style={
        circle,
        fill=Base0D,
        text=Base06
      },
      overlay,
      shift={(5,-0.5)}
    ]
    \node[database] (db) at (0,0) {DB};
    \node[user,right of=db] (u1) {User};
    \node[user,below of=db] (u2) {User};
    \node[user,above of=db] (u3) {User};
    \node[user,left of=db]  (u4) {User};

    \draw[->] (db) -- (u1);
    \draw[->] (db) -- (u2);
    \draw[->] (db) -- (u3);
    \draw[->] (db) -- (u4);
  \end{tikzpicture}
\end{frame}

\begin{frame}[fragile]
  \frametitle{Code Attractiveness}
  \vspace{1cm}
  \epigraph{``Code without tests is broken by design.''}{--- \textup{Jacob Kaplan-Moss}, Django documentation}
\end{frame}

%% \begin{frame}
  \frametitle{Testing Methods}
  \begin{tikzpicture}[overlay]  
  \end{tikzpicture}
\end{frame}

\begin{frame}
  \frametitle{The ``Box'' Approach}
  Describes the point-of-view of the tester.
  Divided into three boxes:
  \begin{itemize}
  \item Black-box testing.
  \item White-box testing.
  \item Grey-box testing.
  \end{itemize}
  Why is this useful? Helps decide on ecapsulation and
  implementation hiding, and with TDD.
\end{frame}

\begin{frame}
  \frametitle{Black-box Testing}
  We don't know anything about internals, only
  the interfaces.

  TODO: Include picture.
\end{frame}

\begin{frame}[fragile]
  \frametitle{Black-box Testing}
  \begin{block}{Class definition}
    \begin{lstlisting}[style=C]
  // Perform Newton-Raphson solve.
  // Solution is bounded by [x1,x2].
  // Initial value is x.
  // Solve tolerance is tol, maximum iterations are
  // max_its.
  template< class FuncT, class T >
  T newton( FuncT func, T x1, T x2, T x,
            T tol, int max_its );
    \end{lstlisting}
  \end{block}
\end{frame}

\begin{frame}[fragile]
  \frametitle{Black-box Testing}
  \begin{block}{Potential tests}
    \begin{itemize}
    \item Check known functions work ($x$, $x^2$, etc).
    \item Check violation of lower/upper bounds.
    \item Check variation of initial x.
    \item Check violation of tolerance.
    \item Check violation of maximum iterations.
    \end{itemize}
  \end{block}
  \pause
  \vspace{1cm}
  {\LARGE\color{Base0A} Can we do better?}
\end{frame}

\begin{frame}
  \frametitle{White-box Testing}
  Have access to application internals.

  TODO: Include picture.
\end{frame}

\begin{frame}[fragile]
  \frametitle{White-box Testing}
  \begin{block}{Class implementation}
    \begin{lstlisting}[style=C]
  template< class FuncT, class T >
  T newton( FuncT func, T x1, T x2, T x,
            T tol, int max_its ) {
    ...
    // Use a scaling to make sure we don't
    // converge simply because x is very small.
    T scale;
    if( x > 0.0 )
      scale = 1.0/x;
    else
      scale = 1.0;
    ...
  }
    \end{lstlisting}
  \end{block}
\end{frame}

\begin{frame}[fragile]
  \frametitle{White-box Testing}
  \begin{block}{More thorough tests}
    \begin{itemize}
    \item Check systems with small scales converge correctly.
    \item Check systems with large scales converge correctly.
    \item Check initial value of 0 works as expected.
    \end{itemize}
  \end{block}
  \pause
  \vspace{1cm}
  {\LARGE\color{Base0A} Perhaps too much effort?}
\end{frame}

\begin{frame}[fragile]
  \frametitle{Grey-box Testing}
  Somewhere in between.
\end{frame}

%% \begin{frame}
  \frametitle{Static Analysis}
  \begin{block}{Compiler errors}
    \begin{itemize}
    \item Parse errors.
    \item Syntax errors.
    \item Missing headers/libraries/symbols.
    \item Fundamental numerical errors.
    \end{itemize}
  \end{block}
  \begin{block}{Static analysis errors}
    \begin{itemize}
    \item Invalid array access.
    \item Scope variable expiration.
    \item Dereferencing null pointer.
    \item Many other logical errors.
    \end{itemize}
  \end{block}
\end{frame}

\begin{frame}
  \frametitle{Static Analysis}
  cppcheck --- {\color{Base0D}\url{http://cppcheck.sourceforge.net}}
  \begin{itemize}
  \item Out of bounds checking.
  \item Check the code for each class.
  \item Checking exception safety.
  \item Memory leaks checking.
  \item Warn if obsolete functions are used.
  \item Check for invalid usage of STL.
  \item Check for uninitialized variables and unused functions.
  \end{itemize}
\end{frame}

\begin{frame}[fragile]
  \frametitle{Static Analysis}
  \begin{example}
    \begin{lstlisting}[style=C]
  // Convert an integer to a string.
  char* scope_var( int val ) {
    char buf[3];
    sprintf( buf, "val=%d", val );
    return buf;
  }

  // Print the integer "10".
  int main() {
    printf( "%s", scope_var( 10 ) );
    return 0;
  }
    \end{lstlisting}
  \end{example}
\end{frame}

\begin{frame}[fragile]
  \frametitle{Static Analysis}
  \begin{example}
    \begin{verbatim}
  cppcheck --enable=all static.c

  [static.c:7]: (error) Pointer to local
    array variable returned.
  [static.c:6]: (error) Buffer is accessed
    out of bounds.
    \end{verbatim}
  \end{example}
\end{frame}

\begin{frame}
  \frametitle{Static Analysis}
  {\Large\color{Base09}What about Python?}

  \vspace{1cm}
  pylint    --- {\color{Base0D}\url{http://www.pylint.org}}
\end{frame}

%% \begin{frame}
  \frametitle{Assertions}
  \begin{tikzpicture}[overlay]
    \node[lit,anchor=north west] (expr) at (2.5,2) {{\Large Is expression true?}};
    \node[bad,anchor=north west] (no) at (5.5,-2) {{\Large Abort}};
    \node[lit,anchor=north west] (yes) at (-0.5,-2) {{\Large Continue}};
    \node[below right=of expr] (nolabel) {No};
    \node[below left=of expr] (yeslabel) {Yes};
    \draw[line width=1pt,draw=Base06,->,>=latex] (expr.south) |- (3,0) -| (yes.north);
    \draw[line width=1pt,draw=Base06,->,>=latex] (expr.south) |- (5,0) -| (no.north);
  \end{tikzpicture}
\end{frame}

\begin{frame}
  \frametitle{Assertions}
  \begin{block}{Advantages}
    \begin{itemize}
    \item Very simple.
    \item Already provided by C/C++ and Python.
    \item Optimised out.
    \end{itemize}
  \end{block}
  \begin{block}{Disadvantages}
    \begin{itemize}
    \item Does not provide much information.
    \item Not a test ``suite'' as such.
    \end{itemize}
  \end{block}
\end{frame}

\begin{frame}[fragile]
  \frametitle{Assertions}
  \begin{example}
    \begin{lstlisting}[style=C]
  char* alloc_a_string( size_t size ) {
    char* p = (char*)malloc( size );

    // What if the allocation fails?
    // Probably a segfault.

    return p;
  }
    \end{lstlisting}
  \end{example}
\end{frame}

\begin{frame}[fragile]
  \frametitle{Assertions}
  \begin{example}
    \begin{lstlisting}[style=C]
  #include <assert.h>

  char* alloc_a_string( size_t size ) {
    char* p = (char*)malloc( size );
    assert( p ); // check allocation
    return p;
  }
    \end{lstlisting}
  \end{example}
\end{frame}

\begin{frame}[fragile]
  \frametitle{Assertions}
  \begin{example}
    \begin{lstlisting}[style=C]
  #ifndef NDEBUG
    #define assert( expr ) <something>
  #else
    #define assert( expr )
  #endif

  // In debug mode.
  gcc -c -o program program.c

  // In optimised mode.
  gcc -DNDEBUG -c -o program program.c
    \end{lstlisting}
  \end{example}
\end{frame}

\begin{frame}[fragile]
  \frametitle{Assertions}
  \begin{example}
    \begin{lstlisting}[style=Py]
  def do_some_asserts(*args, **kw):

    # Must have at least one positional.
    assert len(args) > 0

    # Must have a key of 'hello'.
    assert kw.has('hello')

    # Can also have messages.
    assert kw['hello'] == 'world', 'Wrong!'
    
    \end{lstlisting}
  \end{example}
\end{frame}

%% \begin{frame}
  \frametitle{Testing Levels}
  \begin{tikzpicture}[overlay,shift={(1.5,-2.75)}]
    \visible<1,3->{
      \node[nrm,anchor=north west] at (-1.5,5) {{\Large Unit testing}};
    }
    \visible<2>{
      \node[lit,anchor=north west] at (-1.5,5) {{\Large Unit testing}};
      \node[exp,anchor=north west] at (4.5,5)  {{\small Tests
          the lowest level of the code, focusing on each individual function.}};
      \draw[arr] (4.4,4.5) to (3.6,4.5);
      \node[bb,anchor=north west] at (5,3) {{\small Black-box}};
      \node[wb,anchor=north west] at (5,2) {{\small White-box}};
      \node[gb,anchor=north west] at (5,1) {{\small Grey-box}};
    }

    \visible<-2,4->{
      \node[nrm,anchor=north west] at (-1.5,3.5)   {{\Large Integration testing}};
    }
    \visible<3>{
      \node[lit,anchor=north west] at (-1.5,3.5)   {{\Large Integration testing}};
      \node[exp,anchor=north west] at (4.5,5) {{\small Tests
          the interoperability of separate code pieces.}};
      \draw[arr] (4.4,3) to (3.6,3);
      \node[bb,anchor=north west] at (5,3) {{\small Black-box}};
      \node[gb,anchor=north west] at (5,2) {{\small Grey-box}};
    }

    \visible<-3,5->{
      \node[nrm,anchor=north west] at (-1.5,2) {{\Large System testing}};
    }
    \visible<4>{
      \node[lit,anchor=north west] at (-1.5,2) {{\Large System testing}};
      \node[exp,anchor=north west] at (4.5,5) {{\small Tests the entire system
          working together.}};
      \draw[arr] (4.4,1.5) to (3.6,1.5);
      \node[bb,anchor=north west] at (5,3) {{\small Black-box}};
      \node[gb,anchor=north west] at (5,2) {{\small Grey-box}};
    }

    \visible<-4>{
      \node[nrm,anchor=north west] at (-1.5,0.5)   {{\Large Regression testing}};
    }
    \visible<5>{
      \node[lit,anchor=north west] at (-1.5,0.5)   {{\Large Regression testing}};
      \node[exp,anchor=north west] at (4.5,5) {{\small Ensures previous errors
          don't find their way back in.}};
      \draw[arr] (4.4,0) to (3.6,0);
      \node[bb,anchor=north west] at (5,3) {{\small Black-box}};
    }
  \end{tikzpicture}
\end{frame}

\begin{frame}
  \frametitle{Unit Testing}
  Lowest level of testing.
  Usually directly tests inputs and outputs of functions/methods.
\end{frame}

\begin{frame}
  \frametitle{Unit Testing}
  Things we would like from unit testing:
  \begin{itemize}
  \vspace{0.5cm}
  \item Easy to implement.
  \vspace{0.5cm}
  \item Descriptive failures.
  \vspace{0.5cm}
  \item Minimum boilerplate.
  \vspace{0.5cm}
  \item Flexibility.
  \end{itemize}
\end{frame}

\begin{frame}[fragile]
  \frametitle{Unit Testing}
  \begin{block}{Code to test}
    \begin{lstlisting}[style=C]
  // Complex number implementation.
  struct complex {
    double real;
    double imag;

    complex add( complex const& other ) const;
    complex sub( complex const& other ) const;
    complex div( complex const& other ) const;
    complex mul( complex const& other ) const;
  };
    \end{lstlisting}
  \end{block}
\end{frame}

\begin{frame}[fragile]
  \frametitle{Unit Testing}
  \begin{block}{Test program 1}
    \begin{lstlisting}[style=C]
  int main() {
    complex a{ 1, 0 }, b{ 0, 1 }, c{ 1, 1 };

    assert( a.add( a ) == 2 );
    assert( a.sub( a ) == 0 );
    assert( a.add( b ) == c );
    assert( a.mul( a ) == a );
    assert( c.div( a ) == c );
  }
    \end{lstlisting}
  \end{block}
\end{frame}

\begin{frame}[fragile]
  \frametitle{Unit Testing}
  Things we would like from unit testing:
  \begin{itemize}
  \vspace{0.5cm}
  \item Easy to implement.
  \vspace{0.5cm}
  \item Descriptive failures.
  \vspace{0.5cm}
  \item Minimum boilerplate.
  \vspace{0.5cm}
  \item Flexibility.
  \end{itemize}
  \begin{tikzpicture}[overlay]
    \node[Base0B,scale=2] at (5.25,4) {\checkmark};
    \node[Base08,scale=2] at (5.25,2.8) {$\times$};
    \node[Base0B,scale=2] at (5.25,1.8) {\checkmark};
    \node[Base08,scale=2] at (5.25,0.6) {$\times$};
  \end{tikzpicture}
\end{frame}

\begin{frame}[fragile]
  \frametitle{Unit Testing}
  \begin{block}{Test program 2}
    \begin{lstlisting}[style=C]
  int main() {
    complex a{ 1, 0 }, b{ 0, 1 }, c{ 1, 1 };
    complex c{ 1, 1 };

    if( a.add( a ) != 2 )
      (printf( "addition failed" ), abort());
    if( a.add( b ) != c )
      (printf( "subtraction failed" ), abort());
    if( a.mul( a ) != a )
      (printf( "multiplication failed" ), abort());
    if( c.div( a ) != c )
      (printf( "division failed" ), abort());
  }
    \end{lstlisting}
  \end{block}
\end{frame}

\begin{frame}[fragile]
  \frametitle{Unit Testing}
  Things we would like from unit testing:
  \begin{itemize}
  \vspace{0.5cm}
  \item Easy to implement.
  \vspace{0.5cm}
  \item Descriptive failures.
  \vspace{0.5cm}
  \item Minimum boilerplate.
  \vspace{0.5cm}
  \item Flexibility.
  \end{itemize}
  \begin{tikzpicture}[overlay]
    \node[Base0B,scale=2] at (5.25,4) {\checkmark};
    \node[Base0B,scale=2] at (5.25,2.8) {\checkmark};
    \node[Base08,scale=2] at (5.25,1.8) {$\times$};
    \node[Base08,scale=2] at (5.25,0.6) {$\times$};
  \end{tikzpicture}
\end{frame}

\begin{frame}[fragile]
  \frametitle{Unit Testing}
  We \emph{could} continue adding more, but, as always, someone
  has probably already done this.
\end{frame}

\begin{frame}[fragile]
  \frametitle{C/C++ Unit Testing Frameworks}
  \begin{tabular}{p{3cm}|p{7cm}}
    {\color{Base09}CppUnit} & Excellent framework, very well used/tested, multiple loggers, many test assertions. \\
    \hline
    {\color{Base09}MinUnit} & Possibly the simplest framework available. \\
    \hline
    {\color{Base09}CxxTest} & Essentially no boilerplate, uses Python scripts to detect tests. \\
    \hline
    \only<1>{
      {\color{Base09}Catch} & Personal favorite, combines clever code and rich features.
    }
    \only<2>{
      {\color{Base09}Catch}\hspace{1cm}{\color{Base0B}$\checkmark$} & Personal favorite, combines clever code and rich features.
    }
  \end{tabular}
\end{frame}

\begin{frame}[fragile]
  \frametitle{Unit Testing}
  \begin{block}{Test program 3}
    \begin{lstlisting}[style=C]
  #include <catch.hpp>

  TEST_CASE( "Complex numbers" ) {
    complex a{ 1, 0 }, b{ 0, 1 }, c{ 1, 1 };
    REQUIRE( a.add( a ) == 2 );
    REQUIRE( a.sub( a ) == 0 );
    REQUIRE( a.add( b ) == c );
    REQUIRE( a.mul( a ) == a );
    REQUIRE( c.div( a ) == c );
  }
    \end{lstlisting}
  \end{block}
  \visible<2>{ \tikz[overlay] \draw[Base08,line width=2pt] (0.7,1.2) rectangle (2.3,3.3); }
\end{frame}

\begin{frame}[fragile]
  \frametitle{Successful Results}
  \begin{block}{Minimal output}
    All tests passed (5 assertions in 1 test case)
  \end{block}
  \begin{block}{Verbose output}
    \begin{verbatim}
  complex.cc:55: 
  PASSED:
    REQUIRE( c.div( a ) == c )
  with expansion:
    1 + 1i == 1 + 1i
    \end{verbatim}
  \end{block}
  \visible<2>{
  \begin{tikzpicture}[overlay]
    \draw[Base08,line width=2pt] (2.5,1.7) ellipse (2 and 0.5);
  \end{tikzpicture}
  }
\end{frame}

\begin{frame}
  \frametitle{Test Assertions}
  From {\color{Base0D} CxxTest}:
  \begin{columns}[onlytextwidth]
    \begin{column}{0.4\textwidth}
      \begin{itemize}
      \item {\ttfamily\small TS\_ASSERT}
      \item {\ttfamily\small TS\_ASSERT\_EQUALS}
      \item {\ttfamily\small TS\_ASSERT\_DIFFERS}
      \item {\ttfamily\small TS\_ASSERT\_DELTA}
      \end{itemize}
    \end{column}
    \begin{column}{0.6\textwidth}
      \begin{itemize}
      \item {\ttfamily\small TS\_ASSERT\_LESS\_THAN}
      \item {\ttfamily\small TS\_ASSERT\_LESS\_THAN\_EQUALS}
      \item {\ttfamily\small TS\_ASSERT\_GREATER\_EQUALS}
      \item {\ttfamily\small TS\_ASSERT\_GREATER\_THAN\_EQUALS}
      \end{itemize}
    \end{column}
  \end{columns}
  \vspace{0.5cm}
  From {\color{Base0D} Catch}:
  \begin{itemize}
  \item {\ttfamily\small REQUIRE}
  \end{itemize}
\end{frame}

\begin{frame}[fragile]
  \frametitle{Failed Results}
  \begin{verbatim}
-----------------------------------------------
Complex numbers
-----------------------------------------------
complex.cc:53
...............................................

complex.cc:60: FAILED:
  REQUIRE( c.div( a ) != c )
with expansion:
  1 + 1i != 1 + 1i

===============================================
1 test case - failed (5 assertions - 1 failed)
  \end{verbatim}
\end{frame}

\begin{frame}[fragile]
  \frametitle{Reporters}
  \footnotesize
  \begin{verbatim}
<Catch name="complex">
  <Group>
    <TestCase name="Complex numbers">
      <Expression success="false" filename="complex.cc" line="60">
        <Original>
          c.div( a ) != c
        </Original>
        <Expanded>
          1 + 1i != 1 + 1i
        </Expanded>
      </Expression>
      <OverallResult success="false"/>
    </TestCase>
    <OverallResults successes="4" failures="1"/>
  </Group>
  <OverallResults successes="4" failures="1"/>
</Catch>
  \end{verbatim}
\end{frame}

\begin{frame}[fragile]
  \frametitle{Unit Testing with ``Catch''}
  Things we would like from unit testing:
  \begin{itemize}
  \vspace{0.5cm}
  \item Easy to implement.
  \vspace{0.5cm}
  \item Descriptive failures.
  \vspace{0.5cm}
  \item Minimum boilerplate.
  \vspace{0.5cm}
  \item Flexibility.
  \end{itemize}
  \begin{tikzpicture}[overlay]
    \node[Base0B,scale=2] at (5.25,4) {\checkmark};
    \node[Base0B,scale=2] at (5.25,2.8) {\checkmark};
    \node[Base0B,scale=2] at (5.25,1.8) {\checkmark};
    \node[Base0B,scale=2] at (5.25,0.6) {\checkmark};
  \end{tikzpicture}
\end{frame}

\begin{frame}
  \frametitle{Fixtures}
\end{frame}

\begin{frame}
  \frametitle{Integration Testing}
\end{frame}

\begin{frame}
  \frametitle{System Testing}
\end{frame}

\begin{frame}
  \frametitle{Regression Testing}
\end{frame}

\begin{frame}
  \frametitle{Other Tests}
\end{frame}

%% \begin{frame}
  \frametitle{Automation}
  {\large
  There are two simple ways to achieve test automation:

  \vspace{0.5cm}
  {\color{Base09} Single binary}\\\hspace{0.5cm}Condense all tests into one binary.

  \vspace{0.5cm}
  {\color{Base09} Build system integration}\\\hspace{0.5cm}Augment the build system to run the tests.
  }
  \visible<2->{
    \tikz[overlay] \draw[Base0B,line width=2pt] (-4.4,0.4) ellipse(4.8 and 0.8);
  }
\end{frame}

\begin{frame}[fragile]
  \frametitle{Automation - Makefiles}
  \begin{example}
    \begin{lstlisting}
  test_complex:        $(CC) ...
  test_newton:         $(CC) ...
  test_newton_complex: $(CC) ... 

  check:
       test_complex
       test_newton
       test_newton_complex

  static:
       cppcheck --enable=all ${SRCS}
    \end{lstlisting}
  \end{example}
\end{frame}

\begin{frame}[fragile]
  \frametitle{Automation - Makefiles}
  \begin{block}{Build code}
    \vspace{0.2cm}
    make
    \vspace{0.2cm}
  \end{block}
  \begin{block}{Run tests}
    \vspace{0.2cm}
    make check
    \vspace{0.2cm}
  \end{block}
  \begin{block}{Run static analysis}
    \vspace{0.2cm}
    make static
    \vspace{0.2cm}
  \end{block}
\end{frame}

%% \begin{frame}
  \frametitle{Code Coverage}
  {\Large\color{Base09}What percentage of our source code has been executed by our tests?}
  \begin{block}{Procedure}
  \begin{itemize}
  \item Compile/link with coverage flags.
  \item Run all tests.
  \item Generate readable statistics with one of:
    \begin{itemize}
    \item gcov
    \item lcov
    \end{itemize}
  \item Optionally run genhtml to produce visually pleasant version.
  \end{itemize}
  \end{block}
\end{frame}

\begin{frame}[fragile]
  \frametitle{Code Coverage}
  \begin{example}
    \begin{lstlisting}[style=C]
  int to_test( int x )
  {
    int ii;
    int sum = 0;
    for( ii = 0; ii < 10; ++ii )
    {
      if( x == 0 )
        sum += ii;
      else if( x == 1 )
        sum += 2*ii;
      else
        sum += 3*ii;
    }
    return sum;
  }
    \end{lstlisting}
  \end{example}
\end{frame}

\begin{frame}[fragile]
  \frametitle{Code Coverage}
  \begin{block}{Compile with coverage}
    \vspace{0.2cm}
    \hspace{0.2cm}{\ttfamily gcc -g -O0 {\color{Base09}-fprofile-arcs -ftest-coverage} in.c}
    \vspace{0.2cm}
  \end{block}
  \begin{block}{Run to generate stats}
    \vspace{0.2cm}
    \hspace{0.2cm}{\ttfamily ./a.out}

    \vspace{0.2cm}
    \hspace{1cm} or, more likely,
    \vspace{0.2cm}

    \hspace{0.2cm}{\ttfamily make check}
    \vspace{0.2cm}
  \end{block}
\end{frame}

\begin{frame}[fragile]
  \frametitle{Code Coverage}
  \begin{block}{Run lcov}
    \vspace{0.2cm}
    \hspace{0.2cm}{\ttfamily lcov -c --directory . --output-file results.info}
    \vspace{0.2cm}
  \end{block}
  \vspace{1cm}
  \begin{block}{Generate HTML}
    \vspace{0.2cm}
    \hspace{0.2cm}{\ttfamily genhtml results.info --output-directory html}
    \vspace{0.2cm}
  \end{block}
\end{frame}

\begin{frame}
  \frametitle{Code Coverage}
  \includegraphics[width=\textwidth]{gcov-1.png}
\end{frame}

\begin{frame}
  \frametitle{Code Coverage}
  \includegraphics[width=\textwidth]{gcov-2.png}
\end{frame}


\end{document}
