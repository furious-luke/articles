\title{SED Equations}
\author{
  Luke Hodkinson \\
  Center for Astrophysics and Supercomputing \\
  Swinburne University of Technology \\
  Melbourne, Hawthorn 32000, \underline{Australia}
}
\date{\today}

\documentclass[12pt]{scrartcl}
\usepackage{color}
\usepackage[usenames,dvipsnames]{xcolor}
\usepackage{amsmath}
\usepackage{amsfonts}
\usepackage{amssymb}
\usepackage{siunitx}
\usepackage{listings}
%% \usepackage[scaled]{beramono}
%% \renewcommand*\familydefault{\ttdefault}
%% \usepackage[Tl]{fontenc}

\newcommand{\deriv}[2]{\ensuremath{\frac{\mathrm{d}#1}{\mathrm{d}#2}}}
\newcommand{\sderiv}[2]{\ensuremath{\frac{\mathrm{d}^2#1}{\mathrm{d}#2^2}}}
\newcommand{\dx}[1]{\ensuremath{\,\mathrm{d}#1}}

%% \lstset{
%%   language=Python,
%%   showstringspaces=false,
%%   formfeed=\newpage,
%%   tabsize=4,
%%   basicstyle=\small\ttfamily,
%%   commentstyle=\color{BrickRed}\itshape,
%%   keywordstyle=\color{blue},
%%   stringstyle=\color{OliveGreen},
%%   morekeywords={models, lambda, forms, dict, list, str, import, dir, help,
%%    zip, with, open}
%% }

\begin{document}
\maketitle

\section{Equating Integrands}

In the derivation of the AB magnitudes used in TAO we use the following relationship
between fluxes in wavelengths and frequency,
\[ \int_\nu f_\nu \dx{\nu} = \int_\lambda f_\lambda \dx{\lambda} \; , \]
by transforming the right hand side, giving
\[ \int_\nu f_\nu \dx{\nu} = -\int_\nu f_\lambda \frac{c}{\nu^2} \dx{\lambda} \; , \]
and equating the integrands,
\[ f_\nu = -f_\lambda \frac{c}{\nu^2} \; . \]
In this article I will be showing that the final step, that is equating the
integrands of the transformed integrals, is sound.

In the interest of maintaining relevance to the above stated context, we begin
by defining two equated integrals over different variables,
\begin{eqnarray*}
\int^{x_1}_{x_0} f(x) \dx{x} & = & \int^{h(x_1)}_{h(x_0)} g(y) \,\mathrm{d}y \\
y & = & h(x) \\
x & = & h^{-1}(y)
\end{eqnarray*}
where $f(x)$ and $g(y)$ are arbitrary $C^0$ continuous real-valued functions for $x,y \in \Re$, $x_0$ and $x_1$ are arbitrarily chosen definite integration terminals, $h(x)$ is a real-valued $C^1$ continuous function defining the relationship between $x$ and $y$ and $h^{-1}$ is the inverse of $h$. The first order of business is to transform the RHS to be integrated over the same variable, $x$. We use the derivative of the relationship between $x$ and $y$,
\begin{eqnarray*}
\frac{\,\mathrm{d}y}{\dx{x}} & = & \frac{\,\mathrm{d}h}{\dx{x}} \\
\Rightarrow \,\mathrm{d}y & = & \frac{\,\mathrm{d}h}{\dx{x}} \dx{x}
\end{eqnarray*}
to transform the integrals to
\begin{eqnarray*}
\int^{x_1}_{x_0} f(x) \dx{x} & = & \int^{h(x_1)}_{h(x_0)} g(y) \,\mathrm{d}y \\
& = & \int^{h^{-1}(h(x_1))}_{h^{-1}(h(x_0))} g(h(x)) \frac{\,\mathrm{d}h}{\dx{x}} \dx{x} \\
& = & \int^{x_1}_{x_0} g(h(x)) \frac{\,\mathrm{d}h}{\dx{x}} \dx{x} \\
& = & \int^{x_1}_{x_0} G(x) \dx{x}
\end{eqnarray*}
where we have defined
\[ G(x) = g(h(x)) \frac{\,\mathrm{d}h}{\dx{x}} \; . \]
Now that both integrals are over the same variable and the same domain we may bring them together,
\begin{equation}
\label{all_lhs}
\int^{x_1}_{x_0} f(x) - G(x) \dx{x} = 0 \; .
\end{equation}

We now seek to demonstrate that the above can only by true if
\[ f(x) = G(x) \; . \]
To show that $f(x) = G(x)$ we will use a proof by contradiction; we will show that a combination of $f$ and $G$ that are not equal contradict the requirement that
\[ \int^{x_1}_{x_0} f(x) - G(x) \dx{x} = 0 \; . \]
If $f(x) \ne G(x)$ then there must exist some value of $x$, say $x_e$, such that $f(x_e) \ne G(x_e)$. Now because both $f$ and $G$ are $C^0$ continuous there exists a positive non-zero real-valued parameter $\epsilon$ such that $f(z) < G(z)$ or $f(z) > G(z)$ for all $x_e - \epsilon \le z \le x_e + \epsilon$. With $f(z)$ and $G(z)$ not crossing each other at any point in $z$, $f(z) - G(z)$ will be either be all positive or all negative at all points in $z$. Thus,
\[ \int_{x_e - \epsilon}^{x_e + \epsilon} f(x) - G(x) \dx{x} \ne 0 \; . \]
But this equation is precisely equation \eqref{all_lhs} with the terminals set to $x_0 = x_e - \epsilon$ and $x_1 = x_e + \epsilon$, yet they both claim opposing conclusions, and therein lay the contradiction. Thus, only when $f(x) = G(x)$ can \eqref{all_lhs} be satisfied for all combinations of $x_0$ and $x_1$.

\end{document}
