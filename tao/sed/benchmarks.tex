\title{Magnitude Benchmarks}
\author{
  Luke Hodkinson \\
  Center for Astrophysics and Supercomputing \\
  Swinburne University of Technology \\
  Melbourne, Hawthorn 32000, \underline{Australia}
}
\date{\today}

\documentclass[12pt]{scrartcl}
\usepackage{color}
\usepackage[usenames,dvipsnames]{xcolor}
\usepackage{amsmath}
\usepackage{amsfonts}
\usepackage{amssymb}
\usepackage[scientific-notation=true]{siunitx}
\usepackage{listings}
\usepackage{hyperref}
%% \usepackage[scaled]{beramono}
%% \renewcommand*\familydefault{\ttdefault}
%% \usepackage[Tl]{fontenc}

\newcommand{\deriv}[2]{\ensuremath{\frac{\mathrm{d}#1}{\mathrm{d}#2}}}
\newcommand{\sderiv}[2]{\ensuremath{\frac{\mathrm{d}^2#1}{\mathrm{d}#2^2}}}
\newcommand{\dx}[1]{\ensuremath{\,\mathrm{d}#1}}

%% \lstset{
%%   language=Python,
%%   showstringspaces=false,
%%   formfeed=\newpage,
%%   tabsize=4,
%%   basicstyle=\small\ttfamily,
%%   commentstyle=\color{BrickRed}\itshape,
%%   keywordstyle=\color{blue},
%%   stringstyle=\color{OliveGreen},
%%   morekeywords={models, lambda, forms, dict, list, str, import, dir, help,
%%    zip, with, open}
%% }

\begin{document}
\maketitle

\section{AB Magnitudes}

For each benchmark we begin with the equation for AB magnitudes:
\[ m_{ab} = -2.5\log_{10} \frac{{\displaystyle \int_\lambda f_\lambda
    r \dx{\lambda}}}{{\displaystyle \int_\lambda r \frac{c}{\lambda^2}
    \dx{\lambda}}} - 48.6 \]
For the derivation see that other article I posted to the the TAO
blog. Or I can email it to you.

\section{Unity}

\subsection{Definition}

The simplest benchmark, in which we define things to end up with unity
in as many places as possible.

\begin{itemize}
  \item ${\lambda \in \Re \mid 1 \leq \lambda \le 2}$
  \item $f_\lambda(\lambda) = 1 \, , \quad \forall \lambda$
  \item $r(\lambda) = 1 \, , \quad \forall \lambda$
\end{itemize}

\subsection{Analytical Solution}

This gives the enumerator of the magnitude equation as:
\begin{eqnarray*}
\int_\lambda f_\lambda r \dx{\lambda} & = & \int_\lambda \dx{\lambda}
\\
& = & \int_1^2 \dx{\lambda} \\
& = & 1
\end{eqnarray*}

And the denominator:
\begin{eqnarray*}
\int_\lambda r \frac{c}{\lambda^2} \dx{\lambda} & = & \int_\lambda
\frac{c}{\lambda^2} \dx{\lambda} \\
&= & \int_1^2 \frac{c}{\lambda^2} \dx{\lambda} \\
&= & \left. -\frac{c}{\lambda} \right|_1^2 \\
& = & \frac{c}{2} \si{\angstrom\per\second}
\end{eqnarray*}

Giving an analytical result of:
\begin{eqnarray*}
m_{ab} & = & -2.5\log_{10} \frac{2}{c} - 48.6 \\
& \approx & -3.160523232
\end{eqnarray*}

\subsection{Synthetic Data}

So, we need now to synthesize a dataset that should give us $f_\lambda
= 1$. There are infinitely many ways we could do this and ultimately
it will depend on the software being used, but the approach I used was:
\begin{itemize}
\item I created some fake SSP data with unity as each value. There are
  three age bins each covering $1\mathrm{Gyr}$. There are also 3
  wavelength samples at 1, 1.5 and 2.
\item I created a fake merger tree with three galaxies. Each galaxy
  exists for $1\mathrm{Gyr}$ and carries an SFR of $3.\dot{3}$
  (damn it I've forgotten the units for SFR).
\item I created a fake bandpass filter with unity in each wavelength
  sample.
\end{itemize}
The above produces a rebinned and summed spectra with unity at each
wavelength in the spectrum.

\subsection{Results}

After running through TAO we get:
\[ m_{ab} = -3.160523328 \]
with an error of:
\[ \mathrm{err}(m_{ab}) = \num{-9.5999e-8} \]

\section{Exponential Transmission}

\subsection{Definition}

We will continue to use the unity benchmark, but now we will place an
exponential functionin as transmission function.

\begin{itemize}
  \item $r(\lambda) = \mathrm{e}^{-\lambda}$
\end{itemize}

\subsection{Analytical Solution}

This gives the enumerator of the magnitude equation as:
\begin{eqnarray*}
\int_\lambda f_\lambda r \dx{\lambda} & = & 
  \int_\lambda \mathrm{e}^{-\lambda} \dx{\lambda} \\
& = & \left. -\mathrm{e}^{-\lambda} \right|_1^2 \\
& = & \mathrm{e}^{-1} - \mathrm{e}^{-2} \\
& \approx & 0.232544158
\end{eqnarray*}

And the denominator:
\begin{eqnarray*}
\int_\lambda r \frac{c}{\lambda^2} \dx{\lambda}
& = & \int_\lambda \mathrm{e}^{-\lambda} \frac{c}{\lambda^2}
\dx{\lambda} \\
& \approx & \num{3.889158867e+17}
\end{eqnarray*}

Giving an analytical (sort of) result of:
\begin{eqnarray*}
m_{ab} & = & -2.5\log_{10} \frac{0.232544158}{\num{3.889158867e+17}} - 48.6 \\
& \approx & -3.041624374
\end{eqnarray*}

\subsection{Synthetic Data}

The same as for unity, but modify the transmission function accordingly.

\subsection{Results}

After running through TAO we get:
\[ m_{ab} = -3.041596127 \]
with an error of:
\[ \mathrm{err}(m_{ab}) = \num{2.8247e-5} \]



% \section{Exponential Transmission}

% \subsection{Definition}

% We will continue to use the unity benchmark, but now we will place an
% exponential functionin as transmission function.

% \begin{itemize}
%   \item $r(\lambda) = \mathrm{e}^{-\frac{\lambda-\frac{1}{2}}{2}}$
% \end{itemize}

% \subsection{Analytical Solution}

% This gives the enumerator of the magnitude equation as:
% \begin{eqnarray*}
% \int_\lambda f_\lambda r \dx{\lambda} & = & \int_\lambda
% \mathrm{e}^{-\frac{\lambda-\frac{1}{2}}{2}} \dx{\lambda} \\
% & = & \left. -2\mathrm{e}^{-\frac{\lambda-\frac{1}{2}}{2}} \right|_1^2 \\
% & = & 2\left( \mathrm{e}^{-\frac{1}{4}} - \mathrm{e^{-\frac{3}{4}}}
% \right) \\
% & \approx & 0.612868461
% \end{eqnarray*}

% And the denominator:
% \begin{eqnarray*}
% \int_\lambda r \frac{c}{\lambda^2} \dx{\lambda} & = & \int_\lambda
% \mathrm{e}^{-\frac{\lambda-\frac{1}{2}}{2}} \frac{c}{\lambda^2}
% \dx{\lambda} \\
% & = & \left. c\mathrm{e}^\frac{1}{4}\left(
%   -\frac{\mathrm{e}^{-\frac{\lambda}{2}}}{\lambda} - \frac{1}{2}\left(
%     \mathrm{ln}\lambda +
%     \sum_{n=1}^\infty\frac{\left(-\frac{\lambda}{2}\right)^n}{n\cdot n!}
%   \right) \right) \right|_1^2 \\
% & = & c\mathrm{e}^\frac{1}{4}\left(
%   \left( -\frac{\mathrm{e}^{-1}}{2} - \frac{1}{2}\left( \mathrm{ln}2 +
%     \sum_{n=1}^\infty\frac{\left(-1\right)^n}{n\cdot n!} \right)
% \right) - 
% \left( -\mathrm{e}^{-\frac{1}{2}} - \frac{1}{2}
%     \sum_{n=1}^\infty\frac{\left(-\frac{1}{2}\right)^n}{n\cdot n!}
%   \right) \right) \\
% & = & \frac{c\mathrm{e}^\frac{1}{4}}{2}\left(
%   \left( -\mathrm{e}^{-1} - \left( \mathrm{ln}2 +
%     \sum_{n=1}^\infty\frac{\left(-1\right)^n}{n\cdot n!} \right)
% \right) - 
% \left( -2\mathrm{e}^{-\frac{1}{2}} - 
%     \sum_{n=1}^\infty\frac{\left(-\frac{1}{2}\right)^n}{n\cdot n!}
%   \right) \right) \\
% & = & \frac{c\mathrm{e}^\frac{1}{4}}{2} \left(
%   -\mathrm{e}^{-1} - \mathrm{ln}2 -
%     \sum_{n=1}^\infty\frac{\left(-1\right)^n}{n\cdot n!}
%  +2\mathrm{e}^{-\frac{1}{2}} + 
%     \sum_{n=1}^\infty\frac{\left(-\frac{1}{2}\right)^n}{n\cdot n!}
%   \right) \\
% & = & \frac{c\mathrm{e}^\frac{1}{4}}{2}
% \left(
%   2\mathrm{e}^{-\frac{1}{2}}
%   -\mathrm{e}^{-1}
%   - \mathrm{ln}2
%   + \sum_{n=1}^\infty\left(
%        \frac{\left(-\frac{1}{2}\right)^n - \left(-1\right)^n}{n\cdot n!}
%   \right)
% \right) \\
% & \approx & \num{1.60185875669e18}
% \end{eqnarray*}

% Giving an analytical (sort of) result of:
% \begin{eqnarray*}
% m_{ab} & = & -2.5\log_{10} \frac{0.612868461}{\num{1.601858756e18}} - 48.6 \\
% & \approx & -2.556857633
% \end{eqnarray*}

% \subsection{Synthetic Data}

% The same as for unity, but modify the transmission function accordingly.

% \subsection{Results}

% After running through TAO we get:
% \[ m_{ab} = -3.160523328 \]



% \section{Window}

% \subsection{Definition}

% We will continue to use the unity benchmark, but now we will place a
% window in the transmission function to clip out a range from the
% spectrum.

% \begin{itemize}
%   \item $r(\lambda) = \left\{ \begin{array}{lll}
%           0 &,  1 \leq \lambda \le 1.2 \\
%           1 &, 1.25 \leq \lambda \le 1.75 \\
%           0 &, 1.75 \leq \lambda \le 2
%         \end{array} \right.$
% \end{itemize}

% \subsection{Analytical Solution}

% This gives the enumerator of the magnitude equation as:
% \begin{eqnarray*}
% \int_\lambda f_\lambda r \dx{\lambda} & = & \int_{1.25}^{1.75} \dx{\lambda}
% \\
% & = & \frac{1}{2}
% \end{eqnarray*}

% And the denominator:
% \begin{eqnarray*}
% \int_\lambda r \frac{c}{\lambda^2} \dx{\lambda} & = & \int_\lambda
% \frac{c}{\lambda^2} \dx{\lambda} \\
% &= & \int_{1.25}^{1.75} \frac{c}{\lambda^2} \dx{\lambda} \\
% &= & \left. -\frac{c}{\lambda} \right|_{1.25}^{1.75} \\
% & = & -\frac{c}{1.75} + \frac{c}{1.25} \\
% & = & -\frac{1.25c}{2.1875} + \frac{1.75c}{2.1875} \\
% & = & \frac{0.5c}{2.1875} \\
% & = & \frac{8}{35}c \si{\angstrom\per\second}
% \end{eqnarray*}

% Giving an analytical result of:
% \begin{eqnarray*}
% m_{ab} & = & -2.5\log_{10} \frac{\frac{1}{2}}{\frac{8}{35}c} - 48.6 \\
% & = & -2.5\log_{10} \frac{35}{16c} - 48.6 \\
% & \approx & -3.257818397
% \end{eqnarray*}

% \subsection{Synthetic Data}

% The same as for unity, but modify the transmission function accordingly.

% \subsection{Results}



\end{document}
