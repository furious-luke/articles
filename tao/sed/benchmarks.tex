\title{Magnitude Benchmarks}
\author{
  Luke Hodkinson \\
  Center for Astrophysics and Supercomputing \\
  Swinburne University of Technology \\
  Melbourne, Hawthorn 32000, \underline{Australia}
}
\date{\today}

\documentclass[12pt]{scrartcl}
\usepackage{color}
\usepackage[usenames,dvipsnames]{xcolor}
\usepackage{amsmath}
\usepackage{amsfonts}
\usepackage{amssymb}
\usepackage[scientific-notation=true]{siunitx}
\usepackage{listings}
\usepackage{hyperref}
%% \usepackage[scaled]{beramono}
%% \renewcommand*\familydefault{\ttdefault}
%% \usepackage[Tl]{fontenc}

\newcommand{\deriv}[2]{\ensuremath{\frac{\mathrm{d}#1}{\mathrm{d}#2}}}
\newcommand{\sderiv}[2]{\ensuremath{\frac{\mathrm{d}^2#1}{\mathrm{d}#2^2}}}
\newcommand{\dx}[1]{\ensuremath{\,\mathrm{d}#1}}

%% \lstset{
%%   language=Python,
%%   showstringspaces=false,
%%   formfeed=\newpage,
%%   tabsize=4,
%%   basicstyle=\small\ttfamily,
%%   commentstyle=\color{BrickRed}\itshape,
%%   keywordstyle=\color{blue},
%%   stringstyle=\color{OliveGreen},
%%   morekeywords={models, lambda, forms, dict, list, str, import, dir, help,
%%    zip, with, open}
%% }

\begin{document}
\maketitle

\section{AB Magnitudes}

For each benchmark we begin with the equation for AB magnitudes:
\[ m_{ab} = -2.5\log_{10} \frac{{\displaystyle \int_\lambda f_\lambda
    r \dx{\lambda}}}{{\displaystyle \int_\lambda r \frac{c}{\lambda^2}
    \dx{\lambda}}} - 48.6 \]
For the derivation see that other article I posted to the the TAO
blog. Or I can email it to you.

\section{Synthesized Data}

\subsection{Merger Trees}

I've written a Python utility to generate randomised merger trees. A
suite or parameters controls the depth of each tree and the ranges of
the various attributes each galaxy may possess (star formation rate,
cold gas mass, etc).

Randomised merger trees are useful because running a large number of
them ensures we try a wide variety of possible tree structures and
attribute values.

\subsection{Single Stellar Populations}

The same Python utility as above also generates single stellar
populations, but does so in a very specific manner. SSP data is
generated to ensure that, when combined with merger tree data, will
produce rebinned and summed data that perfectly matches any given
input function. This allows us to generate a combination of data that
will result in the functions later required for benchmarking.

\section{Benchmarks}

The benchmarks are all based on manipulating $f_\lambda(\lambda)$ and
$r(\lambda)$ to produce enumerators and denominators to the magnitude
equation that are either analytically solvable or that can be easily
numerically integrated with high accuracy. This way we can check the
enumerators and denominators directly, along with the final magnitudes.

\subsection{Constant}

\subsubsection{Definition}

The simplest benchmark, in which we define the input functions to be
constants for all $\lambda$:
\begin{eqnarray*}
f_\lambda(\lambda) & = & a \, , \quad \forall \lambda \\
r(\lambda) & = & b \, , \quad \forall \lambda
\end{eqnarray*}
where
\[ \left\{ \lambda \in \Re \mid \lambda_l \leq \lambda \le \lambda_u \right\} \]

\subsection{Analytical Solution}

This gives the enumerator of the magnitude equation as:
\begin{eqnarray*}
\int_\lambda f_\lambda r \dx{\lambda} & = & ab \int_\lambda \dx{\lambda}
\\
& = & ab \int_{\lambda_l}^{\lambda_u} \dx{\lambda} \\
& = & ab\left( \lambda_u - \lambda_l \right)
\end{eqnarray*}

And the denominator:
\begin{eqnarray*}
\int_\lambda r \frac{c}{\lambda^2} \dx{\lambda} & = & b \int_\lambda
\frac{c}{\lambda^2} \dx{\lambda} \\
&= & b \int_{\lambda_l}^{\lambda_u} \frac{c}{\lambda^2} \dx{\lambda} \\
&= & \left. -\frac{bc}{\lambda} \right|_{\lambda_l}^{\lambda_u} \\
& = & bc\left( \frac{1}{\lambda_l} - \frac{1}{\lambda_u} \right)
\end{eqnarray*}

\subsection{Results}

I have generated two sets of results. The first to explore the
percentage as a function of wavelength/bandpass filter resolution, the
second as a function of the wavelength range. It seems the wavelength
range can affect integration accuracy (as I discovered), so I'm
including it for all the benchmarks.

For these results I have set $a=2$ and $b=3$. From figures
~\ref{fig:const_res} and ~\ref{fig:const_wave} it obvious accuracy for
benchmark 1 is excellent, with order $\mathrm{O}(10^{-11})$ and
demonstrating immediate convergence.

\begin{figure}[ht]
  \centerline{\includegraphics{const_res_error.pdf}}
  \caption{Percentage error vs. resolution, holding wavelength range
    constant for benchmark 1.}
  \label{fig:const_res}
\end{figure}

\begin{figure}[ht]
  \centerline{\includegraphics{const_wave_error.pdf}}
  \caption{Percentage error vs. wavelength range, holding resolution
    constant for benchmark 1.}
  \label{fig:const_wave}
\end{figure}


\subsection{Sinusoidal Spectrum}

\subsubsection{Definition}

Similar to benchmark one, but defining:
\begin{eqnarray*}
f_\lambda(\lambda) & = & A\left(\sin(B\lambda) + 1\right) \, , \quad \forall \lambda \\
r(\lambda) & = & C \, , \quad \forall \lambda
\end{eqnarray*}

\subsection{Analytical Solution}

This gives the enumerator of the magnitude equation as:
\begin{eqnarray*}
\int_\lambda f_\lambda r \dx{\lambda}
& = & AC \int_\lambda \sin(B\lambda) + 1 \dx{\lambda} \\
& = & \left. AC\left( -\frac{1}{B}\cos(B\lambda) + \lambda \right)
\right|_{\lambda_l}^{\lambda_u} \\
& = & AC\left( \frac{1}{B}\left( \cos(B\lambda_l) - \cos(B\lambda_u)
  \right) + \lambda_u - \lambda_l \right)
\end{eqnarray*}

And the denominator, which is identical to benchmark 1:
\[ \int_\lambda r \frac{c}{\lambda^2} \dx{\lambda}
 = Cc\left( \frac{1}{\lambda_l} - \frac{1}{\lambda_u} \right) \]

\subsection{Results}

For these results I have set $A=2$, $B=0.01$ and $C=1$. From figures
~\ref{fig:sin_res} and ~\ref{fig:sin_wave} two things are
noticeable. First, the errors are small. Secondly, the errors are
decreasing with increasing resolution and converging at 0. This is to
be expected with a sinusoidal test function, as we are using Gaussian
quadrature for integration.

\begin{figure}[ht]
  \centerline{\includegraphics{sin_res_error.pdf}}
  \caption{Percentage error vs. resolution, holding wavelength range
    constant for benchmark 2.}
  \label{fig:sin_res}
\end{figure}

\begin{figure}[ht]
  \centerline{\includegraphics{sin_wave_error.pdf}}
  \caption{Percentage error vs. wavelength range, holding resolution
    constant, for benchmark 2.}
  \label{fig:sin_wave}
\end{figure}

% \section{Exponential Transmission}

% \subsection{Definition}

% We will continue to use the unity benchmark, but now we will place an
% exponential functionin as transmission function.

% \begin{itemize}
%   \item $r(\lambda) = \mathrm{e}^{-\lambda}$
% \end{itemize}

% \subsection{Analytical Solution}

% This gives the enumerator of the magnitude equation as:
% \begin{eqnarray*}
% \int_\lambda f_\lambda r \dx{\lambda} & = & 
%   \int_\lambda \mathrm{e}^{-\lambda} \dx{\lambda} \\
% & = & \left. -\mathrm{e}^{-\lambda} \right|_1^2 \\
% & = & \mathrm{e}^{-1} - \mathrm{e}^{-2} \\
% & \approx & 0.232544158
% \end{eqnarray*}

% And the denominator:
% \begin{eqnarray*}
% \int_\lambda r \frac{c}{\lambda^2} \dx{\lambda}
% & = & \int_\lambda \mathrm{e}^{-\lambda} \frac{c}{\lambda^2}
% \dx{\lambda} \\
% & \approx & \num{3.889158867e+17}
% \end{eqnarray*}

% Giving an analytical (sort of) result of:
% \begin{eqnarray*}
% m_{ab} & = & -2.5\log_{10} \frac{0.232544158}{\num{3.889158867e+17}} - 48.6 \\
% & \approx & -3.041624374
% \end{eqnarray*}

% \subsection{Synthetic Data}

% The same as for unity, but modify the transmission function accordingly.

% \subsection{Results}

% After running through TAO we get:
% \[ m_{ab} = -3.041596127 \]
% with an error of:
% \[ \mathrm{err}(m_{ab}) = \num{2.8247e-5} \]



% \section{Exponential Transmission}

% \subsection{Definition}

% We will continue to use the unity benchmark, but now we will place an
% exponential functionin as transmission function.

% \begin{itemize}
%   \item $r(\lambda) = \mathrm{e}^{-\frac{\lambda-\frac{1}{2}}{2}}$
% \end{itemize}

% \subsection{Analytical Solution}

% This gives the enumerator of the magnitude equation as:
% \begin{eqnarray*}
% \int_\lambda f_\lambda r \dx{\lambda} & = & \int_\lambda
% \mathrm{e}^{-\frac{\lambda-\frac{1}{2}}{2}} \dx{\lambda} \\
% & = & \left. -2\mathrm{e}^{-\frac{\lambda-\frac{1}{2}}{2}} \right|_1^2 \\
% & = & 2\left( \mathrm{e}^{-\frac{1}{4}} - \mathrm{e^{-\frac{3}{4}}}
% \right) \\
% & \approx & 0.612868461
% \end{eqnarray*}

% And the denominator:
% \begin{eqnarray*}
% \int_\lambda r \frac{c}{\lambda^2} \dx{\lambda} & = & \int_\lambda
% \mathrm{e}^{-\frac{\lambda-\frac{1}{2}}{2}} \frac{c}{\lambda^2}
% \dx{\lambda} \\
% & = & \left. c\mathrm{e}^\frac{1}{4}\left(
%   -\frac{\mathrm{e}^{-\frac{\lambda}{2}}}{\lambda} - \frac{1}{2}\left(
%     \mathrm{ln}\lambda +
%     \sum_{n=1}^\infty\frac{\left(-\frac{\lambda}{2}\right)^n}{n\cdot n!}
%   \right) \right) \right|_1^2 \\
% & = & c\mathrm{e}^\frac{1}{4}\left(
%   \left( -\frac{\mathrm{e}^{-1}}{2} - \frac{1}{2}\left( \mathrm{ln}2 +
%     \sum_{n=1}^\infty\frac{\left(-1\right)^n}{n\cdot n!} \right)
% \right) - 
% \left( -\mathrm{e}^{-\frac{1}{2}} - \frac{1}{2}
%     \sum_{n=1}^\infty\frac{\left(-\frac{1}{2}\right)^n}{n\cdot n!}
%   \right) \right) \\
% & = & \frac{c\mathrm{e}^\frac{1}{4}}{2}\left(
%   \left( -\mathrm{e}^{-1} - \left( \mathrm{ln}2 +
%     \sum_{n=1}^\infty\frac{\left(-1\right)^n}{n\cdot n!} \right)
% \right) - 
% \left( -2\mathrm{e}^{-\frac{1}{2}} - 
%     \sum_{n=1}^\infty\frac{\left(-\frac{1}{2}\right)^n}{n\cdot n!}
%   \right) \right) \\
% & = & \frac{c\mathrm{e}^\frac{1}{4}}{2} \left(
%   -\mathrm{e}^{-1} - \mathrm{ln}2 -
%     \sum_{n=1}^\infty\frac{\left(-1\right)^n}{n\cdot n!}
%  +2\mathrm{e}^{-\frac{1}{2}} + 
%     \sum_{n=1}^\infty\frac{\left(-\frac{1}{2}\right)^n}{n\cdot n!}
%   \right) \\
% & = & \frac{c\mathrm{e}^\frac{1}{4}}{2}
% \left(
%   2\mathrm{e}^{-\frac{1}{2}}
%   -\mathrm{e}^{-1}
%   - \mathrm{ln}2
%   + \sum_{n=1}^\infty\left(
%        \frac{\left(-\frac{1}{2}\right)^n - \left(-1\right)^n}{n\cdot n!}
%   \right)
% \right) \\
% & \approx & \num{1.60185875669e18}
% \end{eqnarray*}

% Giving an analytical (sort of) result of:
% \begin{eqnarray*}
% m_{ab} & = & -2.5\log_{10} \frac{0.612868461}{\num{1.601858756e18}} - 48.6 \\
% & \approx & -2.556857633
% \end{eqnarray*}

% \subsection{Synthetic Data}

% The same as for unity, but modify the transmission function accordingly.

% \subsection{Results}

% After running through TAO we get:
% \[ m_{ab} = -3.160523328 \]



% \section{Window}

% \subsection{Definition}

% We will continue to use the unity benchmark, but now we will place a
% window in the transmission function to clip out a range from the
% spectrum.

% \begin{itemize}
%   \item $r(\lambda) = \left\{ \begin{array}{lll}
%           0 &,  1 \leq \lambda \le 1.2 \\
%           1 &, 1.25 \leq \lambda \le 1.75 \\
%           0 &, 1.75 \leq \lambda \le 2
%         \end{array} \right.$
% \end{itemize}

% \subsection{Analytical Solution}

% This gives the enumerator of the magnitude equation as:
% \begin{eqnarray*}
% \int_\lambda f_\lambda r \dx{\lambda} & = & \int_{1.25}^{1.75} \dx{\lambda}
% \\
% & = & \frac{1}{2}
% \end{eqnarray*}

% And the denominator:
% \begin{eqnarray*}
% \int_\lambda r \frac{c}{\lambda^2} \dx{\lambda} & = & \int_\lambda
% \frac{c}{\lambda^2} \dx{\lambda} \\
% &= & \int_{1.25}^{1.75} \frac{c}{\lambda^2} \dx{\lambda} \\
% &= & \left. -\frac{c}{\lambda} \right|_{1.25}^{1.75} \\
% & = & -\frac{c}{1.75} + \frac{c}{1.25} \\
% & = & -\frac{1.25c}{2.1875} + \frac{1.75c}{2.1875} \\
% & = & \frac{0.5c}{2.1875} \\
% & = & \frac{8}{35}c \si{\angstrom\per\second}
% \end{eqnarray*}

% Giving an analytical result of:
% \begin{eqnarray*}
% m_{ab} & = & -2.5\log_{10} \frac{\frac{1}{2}}{\frac{8}{35}c} - 48.6 \\
% & = & -2.5\log_{10} \frac{35}{16c} - 48.6 \\
% & \approx & -3.257818397
% \end{eqnarray*}

% \subsection{Synthetic Data}

% The same as for unity, but modify the transmission function accordingly.

% \subsection{Results}



\end{document}
