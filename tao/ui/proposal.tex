\title{Partial UI Proposal}
\author{
  Luke Hodkinson \\
  Center for Astrophysics and Supercomputing \\
  Swinburne University of Technology \\
  Melbourne, Hawthorn 32000, \underline{Australia}
}
\date{\today}

\documentclass[12pt]{scrartcl}
\usepackage{color}
\usepackage[usenames,dvipsnames]{xcolor}
\usepackage{amsmath}
\usepackage{amsfonts}
\usepackage{amssymb}
\usepackage{siunitx}
\usepackage{listings}
\usepackage{hyperref}
\usepackage{pgfgantt}
%% \usepackage[scaled]{beramono}
%% \renewcommand*\familydefault{\ttdefault}
%% \usepackage[Tl]{fontenc}

\definecolor{Base00}{HTML}{151515}
\definecolor{Base01}{HTML}{202020}
\definecolor{Base02}{HTML}{303030}
\definecolor{Base03}{HTML}{505050}
\definecolor{Base04}{HTML}{B0B0B0}
\definecolor{Base05}{HTML}{D0D0D0}
\definecolor{Base06}{HTML}{E0E0E0}
\definecolor{Base07}{HTML}{F5F5F5}
\definecolor{Base08}{HTML}{AC4142}
\definecolor{Base09}{HTML}{D28445}
\definecolor{Base0A}{HTML}{F4BF75}
\definecolor{Base0B}{HTML}{90A959}
\definecolor{Base0C}{HTML}{75B5AA}
\definecolor{Base0D}{HTML}{6A9FB5}
\definecolor{Base0E}{HTML}{AA759F}
\definecolor{Base0F}{HTML}{8F5536}

\newcommand{\deriv}[2]{\ensuremath{\frac{\mathrm{d}#1}{\mathrm{d}#2}}}
\newcommand{\sderiv}[2]{\ensuremath{\frac{\mathrm{d}^2#1}{\mathrm{d}#2^2}}}
\newcommand{\dx}[1]{\ensuremath{\,\mathrm{d}#1}}

\lstdefinestyle{C}{
  language=C++,
  %% numbers=left,
  showstringspaces=false,
  formfeed=\newpage,
  tabsize=4,
  basicstyle=\footnotesize\ttfamily,
  commentstyle=\color{Base08}\itshape,
  keywordstyle=\color{Base0D},
  stringstyle=\color{Base0B},
  morekeywords={std, thrust, include, ifndef, define, endif}
  %% morekeywords={models, lambda, forms, dict, list, str, import, dir, help,
  %%  zip, with, open}
}
\lstdefinestyle{Py}{
  language=Python,
  %% numbers=left,
  showstringspaces=false,
  formfeed=\newpage,
  tabsize=4,
  basicstyle=\footnotesize\ttfamily,
  commentstyle=\color{Base08}\itshape,
  keywordstyle=\color{Base0D},
  stringstyle=\color{Base0B},
  morekeywords={assert}
  %% morekeywords={models, lambda, forms, dict, list, str, import, dir, help,
  %%  zip, with, open}
}

\begin{document}
\maketitle

\section{UI Requirements}

\begin{description}
\item[Modularity] Each UI module should be logically independant from the others. This
  does not mean that there can be no interaction between the modules, but the interfaces
  should be considered and clean. A UI developer should be able to create a new module and
  only need to use the prescribed interface mechanisms to extract and output all required
  data.
\item[SM Interface] A more concrete connection between the UI modules and science modules
  is desirable. Taken to the logical conclusion, at least a skeleton, possible even a fully
  functional UI module, should be able to be generated from the implementation of a science
  module.
\item[DRY] Currently the UI system has many repeated implementations of system logic. These
  are often using different libraries (`knockout` for example). Observing a strict ``Don't
  Repeat Yourself'' principle will greatly improve clarity and maintainability.
\item[Simplicity] In order to minimise maintenance overhead we need to refactor and hide as
  much of the ``busy work'' as possible. We should provide a set of tested core routines to perform
  common operations and keep the implementations of the actual UI modules to a minimum.
\end{description}

\section{Proposed Changes}

\begin{description}
\item[Update Django] We are currently using a rather old version of Django. Updating to at least
  1.7 will provide a lot of technical advantages, most coming from code reduction.
\item[Replace UI modules] Implement a new system for representing UI modules. This would be the
  bulk of the work and would result in the greatest value to the overall system. This would allow
  dramatically simpler addition of new modules and maintenance of older modules.
\item[Update DB interface] Currently we are not using Django's migrations scheme for the databases.
  Migrations allow us to modify the front-end DB models in Django and update all actual DB tables
  without ever needing to concern ourselves with the database backend. In addition, some of
  the current Django models are mostly duplicates of others and could easily be refactored using
  Django table inheritance and composition. This part of the work, while important, would not take
  too much time.
\end{description}

\section{Proposed User-Interface Logic}

The presentation of the user interface should be kept separate from the logic of the user
interface. An obvious point of separation is at the Django form level. The Django form embodies
all of the information required by each science module. Beyond the form, the display to the
user can be handled by HTML and a ``templatetags'' library providing a set of helper routines
to easily render the embedded logic from the Django form to, perhaps, a knockout handler.

Presently the creator of a UI module is required to write their own Django form to XML conversion.
In the redesign, only validation is required in the form. Conversion to XML is automatic, but
overridable if necessary.

The additional logic is represented by the use of the Django field validation methods, and by
the addition of two new required methods. The first, ``exports'', dynamically calculates a list
of properties that the module will export to other modules. The second, ``requirements'', returns
a list of properties that needs to be provided by earlier modules. By also providing a mechanism
to query the values of properties from modules earlier in the tree, all direct interaction with
other modules is removed, eliciting a strongly modular environment.

In order to satisfy the DRY principle, we can, for the time being, validate input server-side.
In the future we can look to automatically generate Knockout validation strings and JavaScript
by using Python's builtin AST features. This way we also have a natural fallback to server-side
validation in the case of a JavaScript disabled browser.

From the example Python snippets below, and considering the existing code, it will be obvious
how much of a reduction in both boilerplate and required code will be provided by the proposed
system.

\lstinputlisting[style=Py]{forms_lightcone.py}
\lstinputlisting[style=Py]{forms_sed.py}
\lstinputlisting[style=Py]{forms_filter.py}

\section{The View Component}

The new system would require only very simple views to support the UI modules. All modules
available from a particular view need to be instantiated, evaluated and returned.

\lstinputlisting[style=Py]{views.py}

\section{Template Rendering}

Rendering HTML from templates can be tedious and complicated. Django uses templatetags to greatly
simplify the process by allowing refactoring of common operations. Currently we are not using any
of these features, creating some very complex HTML and JavaScript code. With judicious use of
templatetags we should be able to reduce the burden of the site designer to be concerned only
with the visual appearance of the site.

\section{Task Breakdown}

\begin{figure}[ftbp]
\begin{center}
\begin{ganttchart}[
    hgrid,vgrid,
    x unit=4mm,
    time slot format=isodate
]{2015-01-02}{2015-01-30}
\gantttitlecalendar{year,month,week,weekday} \\
\ganttbar{Preparation}{2015-01-03}{2015-01-03} \\
\ganttlinkedbar{New Django}{2015-01-03}{2015-01-04} \\
\ganttlinkedbar{Update DB}{2015-01-05}{2015-01-06} \\
\ganttlinkedbar{Lightcone}{2015-01-07}{2015-01-12} \\
\ganttlinkedbar{SED}{2015-01-13}{2015-01-15} \\
\ganttlinkedbar{Filter}{2015-01-16}{2015-01-17} \\
\ganttlinkedbar{Selection}{2015-01-18}{2015-01-19} \\
\ganttlinkedbar{SQL}{2015-01-20}{2015-01-22} \\
\ganttlinkedbar{Premade}{2015-01-23}{2015-01-24} \\
\ganttlinkedbar{Views}{2015-01-23}{2015-01-24} \\
\ganttlinkedbar{Templates}{2015-01-25}{2015-01-29} \\
\end{ganttchart}
\end{center}
\caption{Task Breakdown}
\end{figure}

\end{document}
