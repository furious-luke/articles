\title{Partial UI Proposal}
\author{
  Luke Hodkinson \\
  Center for Astrophysics and Supercomputing \\
  Swinburne University of Technology \\
  Melbourne, Hawthorn 32000, \underline{Australia}
}
\date{\today}

\documentclass[12pt]{article}
\usepackage{color}
\usepackage[usenames,dvipsnames]{xcolor}
\usepackage{amsmath}
\usepackage{amsfonts}
\usepackage{amssymb}
\usepackage{siunitx}
\usepackage{listings}
\usepackage{hyperref}
\usepackage{pgfgantt}
\usepackage{microtype}
\usepackage{float}
\usepackage[T1]{fontenc}
\usepackage[scaled]{beramono}

\newcommand\Small{\fontsize{9}{9.2}\selectfont}
\newcommand*\LSTfont{\Small\ttfamily\SetTracking{encoding=*}{-60}\lsstyle}

\definecolor{Base00}{HTML}{151515}
\definecolor{Base01}{HTML}{202020}
\definecolor{Base02}{HTML}{303030}
\definecolor{Base03}{HTML}{505050}
\definecolor{Base04}{HTML}{B0B0B0}
\definecolor{Base05}{HTML}{D0D0D0}
\definecolor{Base06}{HTML}{E0E0E0}
\definecolor{Base07}{HTML}{F5F5F5}
\definecolor{Base08}{HTML}{AC4142}
\definecolor{Base09}{HTML}{D28445}
\definecolor{Base0A}{HTML}{F4BF75}
\definecolor{Base0B}{HTML}{90A959}
\definecolor{Base0C}{HTML}{75B5AA}
\definecolor{Base0D}{HTML}{6A9FB5}
\definecolor{Base0E}{HTML}{AA759F}
\definecolor{Base0F}{HTML}{8F5536}

\newcommand{\deriv}[2]{\ensuremath{\frac{\mathrm{d}#1}{\mathrm{d}#2}}}
\newcommand{\sderiv}[2]{\ensuremath{\frac{\mathrm{d}^2#1}{\mathrm{d}#2^2}}}
\newcommand{\dx}[1]{\ensuremath{\,\mathrm{d}#1}}

\lstdefinestyle{C}{
  language=C++,
  %% numbers=left,
  showstringspaces=false,
  formfeed=\newpage,
  tabsize=4,
  basicstyle=\LSTfont,
  commentstyle=\color{Base08}\itshape,
  keywordstyle=\color{Base0D},
  stringstyle=\color{Base0B},
  morekeywords={std, thrust, include, ifndef, define, endif}
  %% morekeywords={models, lambda, forms, dict, list, str, import, dir, help,
  %%  zip, with, open}
}
\lstdefinestyle{Py}{
  language=Python,
  %% numbers=left,
  showstringspaces=false,
  formfeed=\newpage,
  tabsize=4,
  basicstyle=\LSTfont,
  commentstyle=\color{Base08}\itshape,
  keywordstyle=\color{Base0D},
  stringstyle=\color{Base0B},
  morekeywords={assert}
  %% morekeywords={models, lambda, forms, dict, list, str, import, dir, help,
  %%  zip, with, open}
}

\begin{document}
\maketitle

\section{Why Django}

Instead of doing a full analysis of PHP (and whatever frameworks it has available) vs. Python
and Django I decided to leverage the available resources online. There are quite a lot of
existing comparisons. I'll try to summarize a few of them below.

\begin{description}
\item \href{http://stackoverflow.com/questions/5285035/beginner-friendly-php-or-django}{Beginning PHP or Django} A question on Stack Overflow about choosing PHP or Django. The top answer
recommends Django, highlighting the excellent documentation as a reason.
\item \href{http://onstartups.com/tabid/3339/bid/20493/Why-PHP-Is-Fun-and-Easy-But-Python-Is-Marriage-Material.aspx}{Django in start-ups} The person who made HubSpot talks about why they
chose Django as their framework. Half way down there is a list of dot-points about trade-offs
which is pretty interesting. Also, in the comments there are a few people mentioning that they
are PHP developers actively switching to Python/Django.
\item \href{http://chriskief.com/2012/12/14/time-for-a-switch-php-symfony2-vs-python-django/}{Another comparison} This is another person who analyzed PHP and Django and chose Django.
\item \href{http://blog.sznapka.pl/modern-frameworks-comparison/}{Tech comparison} This is
an interesting one that attempts to look at speed and development costs with various frameworks.
It concludes that there is no clear winner, bearing in mind this is coming from a seasoned
PHP developer. In addition, at the bottom there is a follow-up showing that Django when
configured correctly (mod\_wsgi + cache) is approximately 2x faster than PHP.
\item \href{http://stackoverflow.com/questions/1852876/django-vs-phpframework}{Another Stack Overflow} Another Django vs PHP, also choosing Django and reporting back that they believe
they made the right choice some 9 months later.
\item \href{http://techblog.stickyworld.com/abandoning-php-for-python.html}{Reflecting poorly on PHP} This person again talking about their switch to Django.
\item \href{http://www.reddit.com/r/Python/comments/23jbeq/python\_uses\_why\_python\_vs\_php/}{Longtime PHP user} This person talks about their emerging preference for Python vs PHP. Has
a good list of pros of using Python.
\item \href{https://www.udemy.com/blog/modern-language-wars/}{Language comparison} While
not a comparison of Django directly, this one looks at Python vs PHP. One interesting thing
to note is the popularity comparisons. They show that while PHP has a currently greater
popularity, Python's popularity is increasing quite a lot faster, and also that Python is
currently the single most discussed language. It's also from 2012, so is a little dated.
\end{description}

The reasonably general consensus between developers is that PHP is easy and capable, but
Django provides a structure for good programming practices. As a matter of fact,
I had trouble finding any very positive reviews of PHP in the Django vs. PHP comparisons,
apart from those that suggest PHP is more market favorable, which appears to be true. Having
said that there are people who comment that there are good PHP web frameworks, and I'm sure
this is true, but it does seem that there is a strong turn towards Python/Django.

It seems to me that there is no really clear reason to switch from Django to anything else,
especially just PHP without considering an actual PHP web framework. This is all compounded
by the fact that we already have various sections of Python/Django code that we can pull
across. And, as Darren suggested, we can progress with the
current setup in an incremental approach (beginning with the UI modules and moving on to
the remaining functionality later on).

One last thing, I think it's worth mentioning that my personal preference is Django. I have
no specific issue with PHP at all, but I understand the philosophy that Django uses and can
quickly develop and work with Django projects. To achieve the same level of understanding
using PHP would take time. Sure, I would be able to start programming in a PHP framework
very quickly, but we would be at risk once again of possibly poorly designed code due to
my not having the same appreciation for the design philosophy as I do with Django. Just thought
that should be mentioned, especially if it may be that I will be doing the work on the UI.

\section{UI Requirements}

\begin{description}
\item[Modularity] Each UI module should be logically independent from the others. This
  does not mean that there can be no interaction between the modules, but the interfaces
  should be considered and clean. A UI developer should be able to create a new module and
  only need to use the prescribed interface mechanisms to extract and output all required
  data.
\item[SM Interface] A more concrete connection between the UI modules and science modules
  is desirable. Taken to the logical conclusion, at least a skeleton, possible even a fully
  functional UI module, should be able to be generated from the implementation of a science
  module.
\item[DRY] Currently the UI system has many repeated implementations of system logic. These
  are often using different libraries (`knockout` for example). Observing a strict ``Don't
  Repeat Yourself'' principle will greatly improve clarity and maintainability.
\item[Simplicity] In order to minimize maintenance overhead we need to refactor and hide as
  much of the ``busy work'' as possible. We should provide a set of tested core routines to perform
  common operations and keep the implementations of the actual UI modules to a minimum.
\end{description}

\section{Proposed Changes}

\begin{description}
\item[Update Django] We are currently using a rather old version of Django. Updating to at least
  1.7 will provide a lot of technical advantages, most coming from code reduction.
\item[Replace UI modules] Implement a new system for representing UI modules. This would be the
  bulk of the work and would result in the greatest value to the overall system. This would allow
  dramatically simpler addition of new modules and maintenance of older modules.
\item[Update DB interface] Currently we are not using Django's migrations scheme for the databases.
  Migrations allow us to modify the front-end DB models in Django and update all actual DB tables
  without ever needing to concern ourselves with the database backend. In addition, some of
  the current Django models are mostly duplicates of others and could easily be refactored using
  Django table inheritance and composition. This part of the work, while important, would not take
  too much time.
\end{description}

\section{Proposed User-Interface Logic}

The presentation of the user interface should be kept separate from the logic of the user
interface. An obvious point of separation is at the Django form level. The Django form embodies
all of the information required by each science module. Beyond the form, the display to the
user can be handled by HTML and a ``templatetags'' library providing a set of helper routines
to easily render the embedded logic from the Django form to, perhaps, a knockout handler.

Presently the creator of a UI module is required to write their own Django form to XML conversion.
In the redesign, only validation is required in the form. Conversion to XML is automatic, but
overridable if necessary.

The additional logic is represented by the use of the Django field validation methods, and by
the addition of two new required methods. The first, ``exports'', dynamically calculates a list
of properties that the module will export to other modules. The second, ``requirements'', returns
a list of properties that needs to be provided by earlier modules. By also providing a mechanism
to query the values of properties from modules earlier in the tree, all direct interaction with
other modules is removed, eliciting a strongly modular environment.

In order to satisfy the DRY principle, we can, for the time being, validate input server-side.
In the future we can look to automatically generate Knockout validation strings and JavaScript
by using Python's built-in AST features. This way we also have a natural fallback to server-side
validation in the case of a JavaScript disabled browser.

From the example Python snippets below, and considering the existing code, it will be obvious
how much of a reduction in both boilerplate and required code will be provided by the proposed
system. These snippets don't reflect the whole implementation, they are intended to show the
important concepts in the redesign (i.e. the exports and requirements).

\begin{figure}[H]
\lstinputlisting[style=Py]{forms_lightcone.py}
\caption{Lightcone example form.}
\end{figure}

\begin{figure}[H]
\lstinputlisting[style=Py]{forms_sed.py}
\caption{SED example form.}
\end{figure}

\begin{figure}[H]
\lstinputlisting[style=Py]{forms_filter.py}
\caption{Filter example form.}
\end{figure}

\begin{figure}[H]
\lstinputlisting[style=Py]{forms_image.py}
\caption{Image example form.}
\end{figure}

\section{The View Component}

The new system would require only very simple views to support the UI modules. All modules
available from a particular view need to be instantiated, evaluated and returned.

\begin{figure}[H]
\lstinputlisting[style=Py]{views.py}
\caption{Example TAO view.}
\end{figure}

\section{Template Rendering}

Rendering HTML from templates can be tedious and complicated. Django uses templatetags to greatly
simplify the process by allowing refactoring of common operations. Currently we are not using any
of these features, creating some very complex HTML and JavaScript code. With judicious use of
templatetags we should be able to reduce the burden of the site designer to be concerned only
with the visual appearance of the site.

\section{Task Breakdown}

\begin{figure}[H]
\begin{center}
\begin{ganttchart}[
    hgrid,vgrid,
    x unit=4mm,
    y unit chart=0.55cm,
    time slot format=isodate,
    bar label font=\footnotesize
]{2015-01-01}{2015-01-31}
\gantttitlecalendar{year,month,week} \\
\ganttbar{Prepare repository}{2015-01-03}{2015-01-03} \\
\ganttlinkedbar{Update Django}{2015-01-03}{2015-01-03} \\
\ganttlinkedbar{Make fixture}{2015-01-03}{2015-01-03} \\
\ganttlinkedbar{Redesign models}{2015-01-04}{2015-01-05} \\
\ganttlinkedbar{Migrate DB}{2015-01-05}{2015-01-05} \\
\ganttlinkedbar{Test models}{2015-01-05}{2015-01-06} \\
\ganttlinkedbar{Form design}{2015-01-07}{2015-01-07} \\
\ganttlinkedbar{XML handler}{2015-01-07}{2015-01-09} \\
\ganttlinkedbar{Templatetags}{2015-01-10}{2015-01-10} \\
\ganttlinkedbar{HTML}{2015-01-10}{2015-01-11} \\
\ganttlinkedbar{Testing}{2015-01-11}{2015-01-12} \\
\ganttlinkedbar{SED}{2015-01-13}{2015-01-13} \\
\ganttlinkedbar{SED testing}{2015-01-13}{2015-01-14} \\
\ganttlinkedbar{Filter}{2015-01-15}{2015-01-15} \\
\ganttlinkedbar{Filter testing}{2015-01-15}{2015-01-16} \\
\ganttlinkedbar{Select}{2015-01-17}{2015-01-17} \\
\ganttlinkedbar{Select testing}{2015-01-17}{2015-01-18} \\
\ganttlinkedbar{SQL}{2015-01-19}{2015-01-19} \\
\ganttlinkedbar{SQL testing}{2015-01-19}{2015-01-20} \\
\ganttlinkedbar{Premade}{2015-01-21}{2015-01-21} \\
\ganttlinkedbar{Premade testing}{2015-01-21}{2015-01-22} \\
\ganttlinkedbar{Aux. systems}{2015-01-23}{2015-01-25} \\
\ganttlinkedbar{Testing}{2015-01-26}{2015-01-29} \\
\end{ganttchart}
\end{center}
\caption{Work breakdown.}
\end{figure}

All of the above is on a pretty aggressive schedule. While I think it's possible
to achieve that time-line, it would require 100% of my time for that period, which
is most likely impossible.

\section{Personal Recommendations}

So, I think we have three realistic options:

\begin{enumerate}
\item Continue with Django, rework existing code base incrementally. I can work on this
      immediately.
\item Amr rewrites the code base in a PHP framework when available.
\item Hire an experienced/expert PHP developer to rewrite framework.
\end{enumerate}

There are two points worth mentioning:

\begin{itemize}
\item The reason I've not recommended I proceed with rewriting the code in a PHP framework
      is that I feel my experience with PHP is not sufficient to provide a guarantee-ably 
      well designed system in a timely fashion.
\item I've also not mentioned hiring a Django developer as even an experienced Django developer
      would struggle with interpreting the code we already have. Seeing as I've already had
      the pleasure of deconstructing it I've already got a handle on it.
\end{itemize}

\end{document}
