\title{TAO Priorities 2014}
\author{
  Luke Hodkinson \\
  Center for Astrophysics and Supercomputing \\
  Swinburne University of Technology \\
  Melbourne, Hawthorn 32000, \underline{Australia}
}
\date{\today}

\documentclass[12pt]{scrartcl}
\usepackage{color}
\usepackage[usenames,dvipsnames]{xcolor}
\usepackage{amsmath}
\usepackage{amsfonts}
\usepackage{amssymb}
\usepackage[scientific-notation=true]{siunitx}
\usepackage{listings}
\usepackage{hyperref}
%% \usepackage[scaled]{beramono}
%% \renewcommand*\familydefault{\ttdefault}
%% \usepackage[Tl]{fontenc}

\newcommand{\deriv}[2]{\ensuremath{\frac{\mathrm{d}#1}{\mathrm{d}#2}}}
\newcommand{\sderiv}[2]{\ensuremath{\frac{\mathrm{d}^2#1}{\mathrm{d}#2^2}}}
\newcommand{\dx}[1]{\ensuremath{\,\mathrm{d}#1}}

%% \lstset{
%%   language=Python,
%%   showstringspaces=false,
%%   formfeed=\newpage,
%%   tabsize=4,
%%   basicstyle=\small\ttfamily,
%%   commentstyle=\color{BrickRed}\itshape,
%%   keywordstyle=\color{blue},
%%   stringstyle=\color{OliveGreen},
%%   morekeywords={models, lambda, forms, dict, list, str, import, dir, help,
%%    zip, with, open}
%% }

\begin{document}
\maketitle

\section{Goals}

These are the overarching goals in the order I think they should be
prioritsed.

\begin{enumerate}
  \item Release first open version of TAO (carried over from last
    year).
  \item Ensure satisfactory accuracy of the science modules.
  \item Prevent frontend errors, or any other negative experience from
    the UI.
  \item Extend TAO with new features.
\end{enumerate}

\section{Priorities}

Here I describe a finer grained set of actions to achieve the goals
listed above; these are in order of priority.

I should also say that these priorities are all from a mostly software
engineery perspective, which will likely be quite evident.

\begin{description}
  \item[Import remaining datasets] \hfill \\
    So far we have mini-Millennium (SAGE) and Millennium (SAGE)
    successfully imported into the database with the latest updates to
    both SAGE and TAO. Bolshoi is currently processing and Galacticus
    datasets are being copied to g2.
  \item[Restart TAO and check basic functionality] \hfill \\
    We should do some final testing to check that things are working
    smoothly. In particular, the issue of MPI initialisation failing
    should be corrected now.
  \item[Confirm all internal data present and working] \hfill \\
    By internal data I mean the transmission filters and SSP data. In
    particular, we need to confirm the half-spectrum SSP data is in
    place.
  \item[Install ``init.d'' scripts] \hfill \\
    Should the servers ever go down for any reason we need to be sure
    both the databases and workflow automatically restart; we don't
    want to have to depend on individuals doing it manually.
  \item[Discuss and implement science benchmarks] \hfill \\
    I've made a start on building some science benchmarks to test the
    \emph{accuracy} (emphasied to differentiate from unit tests) of
    TAO, but we need more to be satisfied the numbers we are producing
    are correct.
  \item[Finish unit testing science modules] \hfill \\
    ``Code without tests is broken by design.'' - Jacob Kaplan-Moss,
    Django developer. We already have substantial unit tests, but we
    need to finish them off.
  \item[Develop a system for automatically testing the UI] \hfill \\
    The current system of using Selenium and implementing individual
    unit tests is good, but still lacking in completeness. The
    possible combinations of paths through the UI is very large, and
    an automated system for testing this would be invaluable. Getting
    any kind of error from a web UI can be exceptionally frustrating
    and can be, in my experience, the sole cause of abandoning use of
    a site.
  \item[Extend TAO with new features] \hfill \\
    Not too sure what features should be scheduled next, but I would
    say that ensuring we have a solid foundation to work on is more
    important than adding new things.
  \item[Replace Twitter Bootstrap theme] \hfill \\
    We are currently using a stock standard set of widgets and designs
    for the UI. This is fine, but does not suggest a polished
    site. Having a professional design the theme for our interfaces
    would be a great touch.
  \item[Add basic visualisations to frontend] \hfill \\
    It occurred to me that it wouldn't be too difficult, and could
    enhance users' understanding of what they are asking for from TAO,
    if we added, to the frontend, some generated diagrams of the
    simulation boxes and the lightcone they have requested. It may
    help users to see why they can only select a limited number of
    unique cones, and why images they are building may come back with
    strange ``kinks'' in the middle due to the Gnomonic projection.
\end{description}

\end{document}
