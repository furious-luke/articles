\documentclass{beamer}
%% \documentclass[aspectratio=169]{beamer}

\usepackage[latin1]{inputenc}

% Setup the tikz package for pictures.
\usepackage{tikz}
\usetikzlibrary{calc}
\usetikzlibrary{arrows}

% Set Beamer mode.
\mode<presentation>{
  \usetheme{Oxygen}
}

% Uncomment this if you need to eliminate the
% navigation icons on the bottom right.
%% \setbeamertemplate{navigation symbols}{}

\title[The Cut and Thrust of CUDA]{The Cut and Thrust of CUDA}
\author{Luke Hodkinson}
\institute{
  Center for Astrophysics and Supercomputing \\
  Swinburne University of Technology \\
  Melbourne, Hawthorn 32000, \underline{Australia}
}
\date{\today}

\begin{document}

\frame{\titlepage}

\frame{\frametitle{Table of contents}\tableofcontents}

\AtBeginSection[]
{
  \begin{frame}
    \tableofcontents[currentsection]
  \end{frame}
}

\section{Introduction}

\begin{frame}
  \frametitle{What is Thrust?}
  A general library for using CUDA with C++.
  Uses ``generic programming'' concepts.
  Can make programming CUDA much easier.
\end{frame}

\begin{frame}
  \frametitle{History}
  Where did Thrust come from, who wrote it, when, etc.
\end{frame}

\begin{frame}
  \frametitle{Current Status}
  Actively developed, supported by Nvidia, 240+ active users.
\end{frame}

\section{Results}

\begin{frame}
  \frametitle{Results}
  \framesubtitle{The target problem}
  \begin{itemize}
    \item Conversion of an MPI parallel selection algorithm.
    \item Used to find median of large distributed arrays.
    \item Running with 10,000,000 elements per process.
    \item Written in C++ using generic algorithms (STL).
  \end{itemize}
\end{frame}

\begin{frame}
  \frametitle{Results}
  \framesubtitle{Programming effort}
  \begin{block}{Number of characters modified}
    \centering
    \vspace{1cm}
    {\Huge $24^*$}
    \vspace{1cm}
  \end{block}
  $^*$ {\it Not including build scripts or including headers.}
\end{frame}

\begin{frame}
  \frametitle{Results}
  \framesubtitle{Programming time}
  \begin{block}{Time spent modifying algorithm}
    \centering
    \vspace{1cm}
    {\Huge $\approx 1$ minute$^*$}
    \vspace{1cm}
  \end{block}
  $^*$ {\it Not including build scripts.}
\end{frame}

\begin{frame}
  \frametitle{Results}
  \framesubtitle{Speedup}
  \begin{block}{Measured speedup}
    \centering
    \vspace{1cm}
    {\Huge $122x^*$}
    \vspace{1cm}
  \end{block}
  $^*$ {\it Measured on gSTAR.}
\end{frame}

\begin{frame}
  \frametitle{Results}
  \framesubtitle{Scaling}
  \begin{block}{Weak scaling}
    \centering
    \vspace{1cm}
    \vspace{1cm}
  \end{block}
\end{frame}

\begin{frame}
  \frametitle{Results}
  \framesubtitle{Metrics}
  \begin{block}{Metrics}
    \vspace{1cm}
    \hspace{0.5cm}{\Huge $x5$ per character changed.} \\
    \hspace{0.5cm}{\Huge $x2$ per second spent.}
    \vspace{1cm}
  \end{block}
\end{frame}

\begin{frame}
  \frametitle{Results}
  {\Huge Of course...} \\
  \vspace{.5cm}
  \begin{itemize}
    \item results may vary,
    \item already written in C++,
  \end{itemize}
  \vspace{.5cm}
  which brings us to ...
\end{frame}

\section{C++ and Generics}

\begin{frame}
  \frametitle{Why Am I Talking About C++?}
  Because Thrust is written in C++. \\
  \vspace{.2cm}
  \hspace{.5cm}Why? \\
  \vspace{.2cm}
  \hspace{1cm}Because C++
  \vspace{.1cm}
  \begin{itemize}
    \item allows object-oriented programming,
    \item allows generic programming,
    \item performs as well, if not more so, than C.
  \end{itemize}
\end{frame}

\begin{frame}
  \frametitle{Why Am I Talking About C++?}
  Foreign concepts.
\end{frame}

\begin{frame}
  \frametitle{Differences With C}
  \begin{columns}
    \column{.5\textwidth}
    \begin{block}{C}
    \begin{itemize}
    \item Procedural.
    \end{itemize}
    \end{block}

    \column{.5\textwidth}
    \begin{block}{C++}
    \begin{itemize}
    \item Procedural, object oriented, generic programming.
    \end{itemize}
    \end{block}
  \end{columns}
\end{frame}

\begin{frame}
  \frametitle{Object Oriented Programming}
  Talk about benefits, give some examples. Explain why
  we need it for Thrust and how it helps.
\end{frame}

\begin{frame}
  \frametitle{Generic Programming}
  \framesubtitle{What and why?}
  Explain generally how generics can save us a lot
  of effort by generalising algorithms to various kinds
  of data structures. Separation of algorithms and data.
\end{frame}

\begin{frame}
  \frametitle{Generic Programming}
  \framesubtitle{Templates}
  Achieved in C++ using Templates.
  As an example, show the conversion from an entirely C
  implementation of a function to a C++ templated one.
\end{frame}

\begin{frame}
  \frametitle{Generic Programming}
  \framesubtitle{Data Abstraction and the STL}
  Show examples of a bunch of the STL containers and 
  algorithms.
\end{frame}

\begin{frame}
  \frametitle{Generic Programming}
  \framesubtitle{Iterators}
  Explain iterators.
\end{frame}

\begin{frame}
  \frametitle{Performance}
  \framesubtitle{Zero-overhead principle}
  {\Huge ``You pay for what you use.''}
\end{frame}

\begin{frame}
  \frametitle{Performance}
  \framesubtitle{The Poor Compiler}
  Explain that with templates most of the overhead is
  in the compilation phase. The runtime can be faster.
\end{frame}

\begin{frame}
  \frametitle{Performance}
  \framesubtitle{Code Bloat}
  Talk about this concern and how it doesn't exist.
\end{frame}

\section{Thrust}

\begin{frame}
  \frametitle{What Is It?}
  A C++ set of algorithms and data structures built like
  the STL; with generic programming concepts.
  Give some history.
  \begin{itemize}
    \item History.
    \item Data structures.
    \item Algorithms.
    \item Best practices.
  \end{itemize}
\end{frame}

\begin{frame}
  \frametitle{}
  A C++ set of algorithms and data structures built like
  the STL; with generic programming concepts.
  Give some history.
  \begin{itemize}
    \item History.
    \item Data structures.
    \item Algorithms.
    \item Best practices.
  \end{itemize}
\end{frame}

\begin{frame}
  \frametitle{Applicability}
  Explain that the general operations, transformation and
  reduce, can be applied to a wide range of algorithms.
  Not all, but a wide range. Talk about how the internals of
  these general operations can be modified with functors.
\end{frame}

\begin{frame}
  \frametitle{Applicability}
  List all the different algorithms in the Thrust examples.
\end{frame}

\section{Case Studies}

\begin{frame}
  \frametitle{Median}
  Show the actual code differences.
\end{frame}

\begin{frame}
  \frametitle{Cusp}
  A sparse linear algebra library.
  Include the speedup chart for cusp.
\end{frame}

\begin{frame}
  \frametitle{Lattice Monte-Carlo Simulation}
  Courtesy of Michael Wang.
  \begin{itemize}
    \item 30-40x speedup.
  \end{itemize}
\end{frame}

\section{Conclusion}

\begin{frame}
  \frametitle{Bang for Buck}
  %% \framesubtitle{Some?}
  %% \begin{itemize}
  %%   \item 
  %% \end{itemize}
  Some algorithms are as good as custom kernels.
\end{frame}

\begin{frame}
  \frametitle{Generic Programming}
  %% \framesubtitle{Some?}
  %% \begin{itemize}
  %%   \item 
  %% \end{itemize}
  It's a big plus (plus).
\end{frame}

\begin{frame}
  \frametitle{Long Term Support}
  %% \framesubtitle{Some?}
  %% \begin{itemize}
  %%   \item 
  %% \end{itemize}
  Thrust included with CUDA now.
\end{frame}

\end{document}
