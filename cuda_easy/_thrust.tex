\begin{frame}[fragile]
  \frametitle{Thrust}
  \framesubtitle{Introduction}
  \begin{itemize}
  \item So how can Thrust help us write CUDA capable code?
  \pause
  \item The Thrust library provides replacements for STL containers and
    algorithms.
    \begin{itemize}
    \item \lstinline|vector, transform, copy, count, count_if|
    \end{itemize}
  \pause
  \item And also some new ones.
    \begin{itemize}
    \item \lstinline|reduce, transform_reduce|
    \end{itemize}
  \pause
  \item {\bf They are all hardware optimised.}
  \end{itemize}
\end{frame}

\begin{frame}[fragile]
  \frametitle{Thrust}
  \framesubtitle{Introduction}
  \begin{itemize}
  \item Just like the STL is declared inside the \lstinline|std::| namespace...
  \item ... all thrust symbols are declared inside a \lstinline|thrust::| namespace.
  \item Symbols can be the same name without conflicts.
  \end{itemize}
  \begin{example}
    \begin{lstlisting}
    int main() {
      std::transform( vec.begin(), vec.end(),
                      square<int>() );
      thrust::transform( vec.begin(), vec.end(),
                         square<int>() );
    }
    \end{lstlisting}
  \end{example}
\end{frame}

\begin{frame}[fragile]
  \frametitle{Thrust Vectors}
  \framesubtitle{Device Memory}
  Device memory is distinct from host memory. We need a special
  container class to refer to device memory:
  \begin{block}{Various vectors}
    \begin{lstlisting}
    template< typename T >
    class std::vector<T>;

    template< typename T >
    class thrust::host_vector<T>;

    template< typename T >
    class thrust::device_vector<T>;
    \end{lstlisting}
  \end{block}
\end{frame}

\begin{frame}[fragile]
  \frametitle{Thrust Vectors}
  \framesubtitle{Creating a Vector}
  Device vectors may be constructed with a size...
  \begin{example}
    \begin{lstlisting}
    thrust::device_vector<float> dev_vec( 100 );
    \end{lstlisting}
  \end{example}
  ... or resized afterwards, just like a normal vector.
  \begin{example}
    \begin{lstlisting}
    thrust::device_vector<float> dev_vec;
    dev_vec.resize( 100 );
    \end{lstlisting}
  \end{example}
\end{frame}

\begin{frame}[fragile]
  \frametitle{Thrust Vectors}
  \frametitle{Copying To/From Device}
  Values can be easily transferred from host to device and
  vice-versa.
  \begin{example}
    \begin{lstlisting}
    std::vector<float> host_vec( 100 );
    thrust::device_vector<float> dev_vec( 100 );
    thrust::copy( host_vec.begin(), host_vec.end(),
                  dev_vec.begin() );
    // Do some GPU processing.
    thrust::copy( dev_vec.begin(), dev_vec.end(),
                  host_vec.begin() );
    \end{lstlisting}
  \end{example}
\end{frame}

\begin{frame}
  \frametitle{Thrust Algorithms}
  \framesubtitle{Introduction}
  \begin{itemize}
  \item Thrust's algorithms are its real power.
  \pause
  \item Each algorithm is optimised for every supported hardware type.
  \pause
  \item Like the STL, algorithms are customised with functors.
  \pause
  \item What kinds of algorithms can be optimised for multicore hardware?
  \begin{itemize}
    \item Transformation,
    \item reduction,
    \item sorting,
    \item and others.
  \end{itemize}
  \end{itemize}
\end{frame}

\begin{frame}[fragile]
  \frametitle{Transformation}
  \framesubtitle{Definition}
  Apply a unary operation to each element of a vector and store the
  result in another (or the same) vector.
  \begin{center}
  \begin{tikzpicture}
    \draw[box] (0,0) rectangle(6,1);
    \draw[box] (0,3) rectangle(6,4);
    \foreach \x in {1,...,6}{
      \draw[box] (\x,0) -- (\x,1);
      \draw[box] (\x,3) -- (\x,4);
      \draw[<-] (\x-.5,1.1) -- (\x-.5,2.9);
      \node at (\x-.5,0.5) {$b_\x$};
      \node at (\x-.5,3.5) {$a_\x$};
      \node[op] at (\x-.5,2) {$f(a_\x)$};
    };
  \end{tikzpicture}
  \end{center}
\end{frame}

\begin{frame}[fragile]
  \frametitle{Transformation}
  \framesubtitle{Example}
  \begin{example}
    \begin{lstlisting}
    struct norm {
      __device__
      float operator()( float x ) {
        return sqrt( x*x );
      }
    };

    int main() {
      thrust::device_vector<float> src( 100 ),
                                   dst( 100 );
      // TODO: Initialise "src".
      thrust::transform( src.begin(), src.end(),
                         dst.begin(),
                         norm() );
    }
    \end{lstlisting}
  \end{example}
\end{frame}

\begin{frame}[fragile]
  \frametitle{Reduction}
  \framesubtitle{Definition}
  Apply a binary operation to successive elements of a vector and return
  the scalar result.
  \begin{center}
  \begin{tikzpicture}
    \draw[box] (0,0) rectangle(1,4);
    \foreach \y in {1,...,4}{
      \draw[box] (0,\y) -- (1,\y);
      \node at (.5,\y-.5) {$a_\y$};
    };
    \node at (2,-.5) {$f(a_1,a_2)$};
    \node at (3.5,.2) {$f(f_1,a_3)$};
    \node at (5,.9) {$f(f_2,a_4)$};
    \draw[->] (1.1,.5) -- (1.8,.5) -- (1.8,-.3);
    \draw[->] (1.1,1.5) -- (2.3,1.5) -- (2.3,-.3);
    \draw[->] (1.1,2.5) -- (3.8,2.5) -- (3.8,.4);
    \draw[->] (1.1,3.5) -- (5.3,3.5) -- (5.3,1.1);
    \draw[->] (2.7,-.5) -- (3.3,-.5) -- (3.3,0);
    \draw[->] (4.2,.2) -- (4.8,.2) -- (4.8,.7);
    \draw[->] (5.7,.9) -- (6.3,.9) -- (6.3,1.4);
    \draw[box] (5.8,1.5) rectangle(6.8,2.5);
    \node at (6.3,2) {$b$};
  \end{tikzpicture}
  \end{center}
\end{frame}

\begin{frame}[fragile]
  \frametitle{Reduction}
  \framesubtitle{Example}
  \begin{example}
    \begin{lstlisting}
    int main() {
      thrust::device_vector<float> src( 100 );
      // TODO: Initialise "src".
      thrust::reduce( src.begin(), src.end(),
                      thrust::plus<float>() );
    }
    \end{lstlisting}
  \end{example}
\end{frame}

\begin{frame}[fragile]
  \frametitle{Sorting}
  \framesubtitle{Definition}
  \begin{center}
  \begin{tikzpicture}
    \draw[box] (0,0) rectangle(6,1);
    \draw[box] (0,3) rectangle(6,4);
    \foreach \x in {1,...,6}{
      \draw[box] (\x,0) -- (\x,1);
      \draw[box] (\x,3) -- (\x,4);
      \node at (\x-.5,.5) {$a_\x$};
    };
    \node at (.5,3.5) {$a_3$};
    \node at (1.5,3.5) {$a_1$};
    \node at (2.5,3.5) {$a_5$};
    \node at (3.5,3.5) {$a_4$};
    \node at (4.5,3.5) {$a_2$};
    \node at (5.5,3.5) {$a_6$};
    \draw[->] (.5,2.9) -- (2.5,1.1);
    \draw[->] (1.5,2.9) -- (.5,1.1);
    \draw[->] (2.5,2.9) -- (4.5,1.1);
    \draw[->] (3.5,2.9) -- (3.5,1.1);
    \draw[->] (4.5,2.9) -- (1.5,1.1);
    \draw[->] (5.5,2.9) -- (5.5,1.1);
  \end{tikzpicture}
  \end{center}
\end{frame}

\begin{frame}[fragile]
  \frametitle{Sorting}
  \framesubtitle{Example}
  \begin{example}
    \begin{lstlisting}
    int main() {
      thrust::device_vector<float> src( 100 );
      // TODO: Initialise "src".
      thrust::sort( src.begin(), src.end(),
                    thrust::greater<float>() );
    }
    \end{lstlisting}
  \end{example}
\end{frame}

\begin{frame}
  \frametitle{Application}
  You would be forgiven for thinking that these 3 algorithms don't
  cover much. However, many more complex algorithms may be decomposed
  into these algorithmic methods. \\
  \vspace{.4cm}
  Combined with the range of thrust functors and your own custom functors
  many different complex algorithms can be implemented.
\end{frame}

\begin{frame}
  \frametitle{Application}
  \framesubtitle{Examples}
  \begin{example}
    \begin{columns}
      \begin{column}{.4\textwidth}
        \begin{itemize}
        \item Discrete Voronoi
        \item Bucket sort
        \item Bounding box
        \item Histograms
        \item Lexicographical sort
        \end{itemize}
      \end{column}
      \begin{column}{.5\textwidth}
        \begin{itemize}
        \item Monte carlo disjoint sequences
        \item Padded grid reduction
        \item Run-length encoding
        \item SAXPY
        \item Set operations
        \end{itemize}
      \end{column}
    \end{columns}
  \end{example}
\end{frame}

\begin{frame}
  \frametitle{Application}
  \framesubtitle{Strategy}
  \begin{enumerate}
  \item Does your algorithm subscribe to a fundamental algorithm (transform, reduce, sort)? Done.
  \pause
  \item Can your algorithm be decomposed into simpler components? Goto 1 for each component.
  \pause
  \item Can't decompose any further and doesn't match a fundamental algorithm? Write your own kernel.
  \end{enumerate}
\end{frame}

\begin{frame}[fragile]
  \frametitle{Advanced Iterators}
  \framesubtitle{Overview}
  There are other powerful features of Thrust, designed to improve
  flexibility and performance.
  \begin{itemize}
  \item \lstinline|thrust::constant_iterator|,
  \item \lstinline|thrust::counting_iterator|,
  \item \lstinline|thrust::transform_iterator| and
  \item \lstinline|thrust::zip_iterator|.
  \end{itemize}
\end{frame}
