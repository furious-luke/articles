\title{Yay! Equations!}

\documentclass[12pt]{scrartcl}
\usepackage{color}
\usepackage[usenames,dvipsnames]{xcolor}
\usepackage{amsmath}
\usepackage{amsfonts}
\usepackage{amssymb}
\usepackage{siunitx}
\usepackage{listings}
\usepackage{hyperref}
%% \usepackage[scaled]{beramono}
%% \renewcommand*\familydefault{\ttdefault}
%% \usepackage[Tl]{fontenc}

% Setup the tikz package for pictures.
\usepackage{tikz}
\usetikzlibrary{calc}
\usetikzlibrary{arrows}
\usetikzlibrary{shapes,decorations.pathmorphing}

\newcommand{\deriv}[2]{\ensuremath{\frac{\mathrm{d}#1}{\mathrm{d}#2}}}
\newcommand{\sderiv}[2]{\ensuremath{\frac{\mathrm{d}^2#1}{\mathrm{d}#2^2}}}
\newcommand{\dx}[1]{\ensuremath{\,\mathrm{d}#1}}

\begin{document}
\maketitle

\section{Fourier Transformation}

This one is my favorite equation, so it comes first. I'll start by trying
to explain a bit about what it does. I think it might actually be kind
of a reasonable option for your present, as the Fourier transformation is
used all the time for sound related operations. For example, one if its
foremost uses is to convert a series of sound wave data from a time series
to a frequency series. So, what does that actually mean. Well, if you were
to plot a sound wave, much like what you're doing with the first part
of the present, you would have it in a time series format. Along the x-axis
is time, and along the y-axis is the amplitude of the wave (how compressed/
extended the air is in that part of the wave).

Now, if you were to then run that data through the Fourier transform you'd
end up with a plot of the frequency content of that sound. That would mean
that the x-axis would then be frequency instead of time, and the y-axis would
be the amount of that frequency involved in the sound.

Besides the equation being very elegant, it's a fascinating concept to me;
that from a set of data providing no obvious information about frequency we
can deduce directly how much of any frequency you may care to consider exists
within that data.

The mechanism by which it works is also interesting, I think. It uses the
concept of basis functions in functional spaces to decompose the wave
into parts, then shift those parts to a new basis in a frequency domain. Without
doing a 3 hour lecture it's a bit much to lay on someone (sorry!), but
I think it's pretty cool.

Enough of all that, here is what it looks like:
\[ \hat{f}(\nu) = \int_{-\infty}^{\infty}f(t)e^{-2\pi it\nu}\dx{t} \; . \]
Here we have:
\begin{eqnarray*}
\hat{f} & = & \textrm{the wave function in Fourier (frequency) space,} \\
\nu & = & \textrm{frequency,} \\
f & = & \textrm{the wave function in time space,} \\
t & = & \textrm{time,} \\
e & = & \textrm{the natural number,} \\
\pi & = & \textrm{I'm sure you don't need explanation of this guy,} \\
i & = & \textrm{the imaginary number (also cool).}
\end{eqnarray*}
The big looking ``S'' thing at the immediate right of the = sign is an integral
symbol, it goes with the ``dt'' bit at the end, don't worry too much
about that, it's a part of calculus and might be a bit much to cover here.

\section{Euler's Identity}

Okay, so if it's elegance you're after it doesn't get any better than this:
\[ e^{i\pi} + 1 = 0 \; . \]
It's so simple, but involves a whole lot of great stuff, and says something
lovely about the natural number ($e$), geometric functions ($sin$ and $cos$)
and the unit circle.

This is a pretty good one for the person that enjoys the universal constants,
like $e$, $\pi$, and even good old $0$ and $1$. Also, once again, we have that
imaginary number in there, which is always fun.

Furthermore, here is some stuff people (mathematicians) have said about it:
\begin{itemize}
  \item the gold standard for mathematical beauty,
  \item it is absolutely paradoxical; we cannot understand it, and we don't know what it means, but we have proved it, and therefore we know it must be the truth
  \item Like a Shakespearean sonnet that captures the very essence of love, or a painting that brings out the beauty of the human form that is far more than just skin deep, Euler's equation reaches down into the very depths of existence
  \item the most famous formula in all mathematics
\end{itemize}

So, you can see that this is pretty damn popular!

\section{Zeta Function Regularisation and an Infinite Sum}

This one is also AMAZING, as in mind blown kind of stuff. As a matter
of fact, you may not even believe it yourself. Anyway, let's have a look.

Say I asked you to sum up the natural numbers up to a certain limit. So,
something like this:
\[ 1 + 2 + 3 + ... + N \; . \]
What could you say about the answer? Well, first we could definitely say
it would be positive. And also, it would probably be pretty big. And, for
almost all possible values of $N$ you would be right.

But, what if $N = \infty$? What would you say then? The obvious answer is
that the sum might be $\infty$ itself. I mean, that kind of makes sense,
right? Summing up a whole bunch of positive numbers up to and including
infinity; it should be infinity.

Wrong! And you'll never guess what the actual answer is. It's $-\frac{1}{12}$.
Yep. A negative number. Mind equals blown. You might be tempted to say that
it's just a mathematical trick, and doesn't actually mean anything. However this
result is actually used in cutting edge physics, so may indeed have real
physical meaning. The mathematics behind it aren't even
that complicated. Next time I see you I'll show you how it's done. It's very
cool.

Anyway, the actual equation would look like this:
\[ \sum_{i=1}^{\infty}i = -\frac{1}{12} \; . \]

\section{Time-dependent Schr\"{o}dinger Equation}

Okay, last one. This one is the basis for quantum mechanics, and while it's
not really that elegant in form, it provides a foundation for some of the
most counter-intuitive results to have come out of mathematics and physics.
I like this equation because it underlies the fact that the universe behaves
in ways that are impossible to understand without mathematics to assist us.

This is what it looks like:
\[ i\hbar\frac{\partial}{\partial t}\Psi(\mathbf{r},t) = \left[\frac{-\hbar^2}{2\mu}\nabla^2 + 
   V(\mathbf{r},t)\right]\Psi(\mathbf{r},t) \; . \]
I'm afraid going in to detail about it would take some time. I'd be happy
to provide some more details if you think you'd like to use it and need to know
a bit more about it.

\end{document}
