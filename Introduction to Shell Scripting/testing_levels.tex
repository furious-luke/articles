\begin{frame}
  \frametitle{Testing Levels}
  \begin{tikzpicture}[overlay]

    \node[bkg,anchor=north west,text width=3.2cm,text depth=6cm,label=above:{\color{Base04}Static}] at (-0.5,2.3) {};
    \node[bkg,anchor=north west,text width=6.8cm,text depth=6cm,label=above:{\color{Base04}Dynamic}] at (3.8,2.3) {};

    \node[nrm2,short,anchor=north west] at (-0.2,1.9) {Reviews};
    \node[nrm2,short,anchor=north west] at (-0.2,0.7) {Walkthroughs};
    \node[nrm2,short,anchor=north west] at (-0.2,-0.5) {Inspections};
    \node[nrm2,short,anchor=north west] at (-0.2,-1.7) {IDE (syntax)};
    \node[nrm2,long,anchor=north west] (F) at (4.1,1.9) {Functional};
    \node[nrm2,long,anchor=north west] (N) at (4.1,-1.6) {Non-functional};

    \coordinate (o) at (6,0.7);
    \coordinate (p) at (6,-2.8);
    \coordinate (g) at (7.52,-3);

    \node[lit,tiny,anchor=north west] (a) at (4.1,0.5)  {Unit};
    \node[lit,tiny,anchor=north west] (b) at (5.1,-0.4) {Integration};
    \node[lit,tiny,anchor=north west] (c) at (6.72,0.5)  {System};
    \node[lit,tiny,anchor=north west] (d) at (7.8,-0.4) {Regression};
    \node[nrm2,tiny,anchor=north west] (e) at (9.4,0.5)  {Sanity};

    \node[nrm2,short,anchor=north west] at (-0.2,-2.9) {Analysis};

    \node[nrm2,tiny,anchor=north west] (f) at (4.1,-3)  {Performance};
    \node[nrm2,tiny,anchor=north west]     at (6.95,-3) {Usability};
    \node[nrm2,tiny,anchor=north west] (h) at (9.18,-3)  {Security};

    \path[draw,color=Base06,-latex] (F) |- (o) -| (a);
    \path[draw,color=Base06,-latex] (F) |- (o) -| (b);
    \path[draw,color=Base06,-latex] (F) |- (o) -| (c);
    \path[draw,color=Base06,-latex] (F) |- (o) -| (d);
    \path[draw,color=Base06,-latex] (F) |- (o) -| (e);

    \path[draw,color=Base06,-latex] (N) |- (p) -| (f);
    \path[draw,color=Base06,-latex] (N) |- (p) -| (g);
    \path[draw,color=Base06,-latex] (N) |- (p) -| (h);

  \end{tikzpicture}
\end{frame}

\begin{frame}
  \frametitle{Testing Levels}
  \begin{tikzpicture}[overlay,shift={(1.5,-2.75)}]
    \visible<1,3->{
      \node[nrm,anchor=north west] at (-1.5,5) {{\Large Unit testing}};
    }
    \visible<2>{
      \node[lit,anchor=north west] at (-1.5,5) {{\Large Unit testing}};
      \node[exp,anchor=north west] at (4.5,5)  {{\small Tests
          the lowest level of the code, focusing on each individual function.}};
      \draw[arr] (4.4,4.5) to (3.6,4.5);
      \node[bb,anchor=north west] at (5,3) {{\small Black-box}};
      \node[wb,anchor=north west] at (5,2) {{\small White-box}};
      \node[gb,anchor=north west] at (5,1) {{\small Grey-box}};
    }

    \visible<-2,4->{
      \node[nrm,anchor=north west] at (-1.5,3.5)   {{\Large Integration testing}};
    }
    \visible<3>{
      \node[lit,anchor=north west] at (-1.5,3.5)   {{\Large Integration testing}};
      \node[exp,anchor=north west] at (4.5,5) {{\small Tests
          the interoperability of separate code pieces.}};
      \draw[arr] (4.4,3) to (3.6,3);
      \node[bb,anchor=north west] at (5,3) {{\small Black-box}};
      \node[gb,anchor=north west] at (5,2) {{\small Grey-box}};
    }

    \visible<-3,5->{
      \node[nrm,anchor=north west] at (-1.5,2) {{\Large System testing}};
    }
    \visible<4>{
      \node[lit,anchor=north west] at (-1.5,2) {{\Large System testing}};
      \node[exp,anchor=north west] at (4.5,5) {{\small Tests the entire system
          working together.}};
      \draw[arr] (4.4,1.5) to (3.6,1.5);
      \node[bb,anchor=north west] at (5,3) {{\small Black-box}};
      \node[gb,anchor=north west] at (5,2) {{\small Grey-box}};
    }

    \visible<-4>{
      \node[nrm,anchor=north west] at (-1.5,0.5)   {{\Large Regression testing}};
    }
    \visible<5>{
      \node[lit,anchor=north west] at (-1.5,0.5)   {{\Large Regression testing}};
      \node[exp,anchor=north west] at (4.5,5) {{\small Ensures previous errors
          don't find their way back in.}};
      \draw[arr] (4.4,0) to (3.6,0);
      \node[bb,anchor=north west] at (5,3) {{\small Black-box}};
    }
  \end{tikzpicture}
\end{frame}

\begin{frame}
  \frametitle{Unit Testing}
  \begin{tikzpicture}[overlay]

    \node[bkg,anchor=north west,text width=3.2cm,text depth=6cm,label=above:{\color{Base04}Static}] at (-0.5,2.3) {};
    \node[bkg,anchor=north west,text width=6.8cm,text depth=6cm,label=above:{\color{Base04}Dynamic}] at (3.8,2.3) {};

    \node[nrm2,short,anchor=north west] at (-0.2,1.9) {Reviews};
    \node[nrm2,short,anchor=north west] at (-0.2,0.7) {Walkthroughs};
    \node[nrm2,short,anchor=north west] at (-0.2,-0.5) {Inspections};
    \node[nrm2,short,anchor=north west] at (-0.2,-1.7) {IDE (syntax)};
    \node[nrm2,long,anchor=north west] (F) at (4.1,1.9) {Functional};
    \node[nrm2,long,anchor=north west] (N) at (4.1,-1.6) {Non-functional};

    \coordinate (o) at (6,0.7);
    \coordinate (p) at (6,-2.8);
    \coordinate (g) at (7.52,-3);

    \node[lit,tiny,anchor=north west] (a) at (4.1,0.5)  {Unit};
    \node[nrm2,tiny,anchor=north west] (b) at (5.1,-0.4) {Integration};
    \node[nrm2,tiny,anchor=north west] (c) at (6.72,0.5)  {System};
    \node[nrm2,tiny,anchor=north west] (d) at (7.8,-0.4) {Regression};
    \node[nrm2,tiny,anchor=north west] (e) at (9.4,0.5)  {Sanity};

    \node[nrm2,short,anchor=north west] at (-0.2,-2.9) {Analysis};

    \node[nrm2,tiny,anchor=north west] (f) at (4.1,-3)  {Performance};
    \node[nrm2,tiny,anchor=north west]     at (6.95,-3) {Usability};
    \node[nrm2,tiny,anchor=north west] (h) at (9.18,-3)  {Security};

    \path[draw,color=Base06,-latex] (F) |- (o) -| (a);
    \path[draw,color=Base06,-latex] (F) |- (o) -| (b);
    \path[draw,color=Base06,-latex] (F) |- (o) -| (c);
    \path[draw,color=Base06,-latex] (F) |- (o) -| (d);
    \path[draw,color=Base06,-latex] (F) |- (o) -| (e);

    \path[draw,color=Base06,-latex] (N) |- (p) -| (f);
    \path[draw,color=Base06,-latex] (N) |- (p) -| (g);
    \path[draw,color=Base06,-latex] (N) |- (p) -| (h);

  \end{tikzpicture}
\end{frame}

\begin{frame}
  \frametitle{Unit Testing}
  \visible<2->{
    \vspace{4cm}
    {\Large\color{Base09}Do standalone functions/methods do what they are supposed to?}
  }
  \begin{tikzpicture}[overlay,shift={(-3,3)}]
    \node[class] (A) at (3,2)   {\textbf{Class A}\nodepart{second}method\_1};
    \node[class] (B) at (7,2)   {\textbf{Class B}\nodepart{second}method\_2};
    \node[class] (C) at (0.5,0) {\textbf{Class C}\nodepart{second}method\_3};
    \coordinate  (O) at (4.5,-0.5);
    \path[draw,Base06,<-,>=open diamond,thick] (C) -- (O);
    \path[draw,Base06,->,>=latex,thick] (O) -- (A);
    \path[draw,Base06,->,>=latex,thick] (O) -| (B);
    \visible<3->{
      \node[Base08,anchor=north west,scale=8] at (-0.3,1.5) {$\times$};
      \draw[Base0B,anchor=north west,line width=2pt] (4.5,1.5) ellipse(1.8 and 0.8);
      \draw[Base0B,anchor=north west,line width=2pt] (8.5,1.5) ellipse(1.8 and 0.8);
    }
  \end{tikzpicture}
\end{frame}

\begin{frame}
  \frametitle{Unit Testing}
  Things we would like from unit testing:
  \begin{itemize}
  \vspace{0.5cm}
  \item Easy to implement.
  \vspace{0.5cm}
  \item Descriptive failures.
  \vspace{0.5cm}
  \item Minimum boilerplate.
  \vspace{0.5cm}
  \item Flexibility.
  \end{itemize}
\end{frame}

\begin{frame}[fragile]
  \frametitle{Unit Testing}
  \begin{block}{Code to test}
    \begin{lstlisting}[style=C]
  // Complex number implementation.
  struct complex {
    double real;
    double imag;

    complex add( complex const& other ) const;
    complex sub( complex const& other ) const;
    complex div( complex const& other ) const;
    complex mul( complex const& other ) const;
  };
    \end{lstlisting}
  \end{block}
\end{frame}

\begin{frame}[fragile]
  \frametitle{Unit Testing}
  \begin{block}{Test program 1}
    \begin{lstlisting}[style=C]
  int main() {
    complex a{ 1, 0 }, b{ 0, 1 }, c{ 1, 1 };

    assert( a.add( a ) == 2 );
    assert( a.sub( a ) == 0 );
    assert( a.add( b ) == c );
    assert( a.mul( a ) == a );
    assert( c.div( a ) == c );
  }
    \end{lstlisting}
  \end{block}
\end{frame}

\begin{frame}[fragile]
  \frametitle{Unit Testing}
  Things we would like from unit testing:
  \begin{itemize}
  \vspace{0.5cm}
  \item Easy to implement.
  \vspace{0.5cm}
  \item Descriptive failures.
  \vspace{0.5cm}
  \item Minimum boilerplate.
  \vspace{0.5cm}
  \item Flexibility.
  \end{itemize}
  \begin{tikzpicture}[overlay]
    \node[Base0B,scale=2] at (5.25,4) {\checkmark};
    \node[Base08,scale=2] at (5.25,2.8) {$\times$};
    \node[Base0B,scale=2] at (5.25,1.8) {\checkmark};
    \node[Base08,scale=2] at (5.25,0.6) {$\times$};
  \end{tikzpicture}
\end{frame}

\begin{frame}[fragile]
  \frametitle{Unit Testing}
  \begin{block}{Test program 2}
    \begin{lstlisting}[style=C]
  int main() {
    complex a{ 1, 0 }, b{ 0, 1 }, c{ 1, 1 };

    if( a.add( a ) != 2 )
      (printf( "addition failed" ), abort());
    if( a.add( b ) != c )
      (printf( "subtraction failed" ), abort());
    if( a.mul( a ) != a )
      (printf( "multiplication failed" ), abort());
    if( c.div( a ) != c )
      (printf( "division failed" ), abort());
  }
    \end{lstlisting}
  \end{block}
\end{frame}

\begin{frame}[fragile]
  \frametitle{Unit Testing}
  Things we would like from unit testing:
  \begin{itemize}
  \vspace{0.5cm}
  \item Easy to implement.
  \vspace{0.5cm}
  \item Descriptive failures.
  \vspace{0.5cm}
  \item Minimum boilerplate.
  \vspace{0.5cm}
  \item Flexibility.
  \end{itemize}
  \begin{tikzpicture}[overlay]
    \node[Base0B,scale=2] at (5.25,4) {\checkmark};
    \node[Base0B,scale=2] at (5.25,2.8) {\checkmark};
    \node[Base08,scale=2] at (5.25,1.8) {$\times$};
    \node[Base08,scale=2] at (5.25,0.6) {$\times$};
  \end{tikzpicture}
\end{frame}

\begin{frame}[fragile]
  \frametitle{Unit Testing}
  We {\color{Base09}\emph{could}} continue adding more, but, as always, someone
  has already done this for us.
\end{frame}

\begin{frame}[fragile]
  \frametitle{C/C++ Unit Testing Frameworks}
  \begin{tabular}{p{3cm}|p{7cm}}
    {\color{Base09}CppUnit} & Excellent framework, very well used/tested, multiple loggers, many test assertions.\newline {\color{Base0D}\url{http://cppunit.sourceforge.net}} \vspace{0.1cm} \\
    \hline
    {\color{Base09}MinUnit} & Possibly the simplest framework available.\vspace{0.1cm} {\color{Base0D}\url{http://github.com/siu/minunit}} \vspace{0.1cm} \\
    \hline
    {\color{Base09}CxxTest} & Essentially no boilerplate, uses Python scripts to detect tests.\newline {\color{Base0D}\url{http://cxxtest.com}} \vspace{0.1cm} \\
    \hline
    \only<1>{
      {\color{Base09}Catch} & Personal favorite, combines clever code and rich features.\newline {\color{Base0D}\url{http://github.com/philsquared/catch}}
    }
    \only<2>{
      {\color{Base09}Catch}\hspace{1cm}{\color{Base0B}$\checkmark$} & Personal favorite, combines clever code and rich features.\newline {\color{Base0D}\url{http://github.com/philsquared/catch}}
    }
  \end{tabular}
\end{frame}

\begin{frame}[fragile]
  \frametitle{Unit Testing}
  \begin{block}{Test program 3}
    \begin{lstlisting}[style=C]
  #include <catch.hpp>

  TEST_CASE( "Complex numbers" ) {
    complex a{ 1, 0 }, b{ 0, 1 }, c{ 1, 1 };
    REQUIRE( a.add( a ) == 2 );
    REQUIRE( a.sub( a ) == 0 );
    REQUIRE( a.add( b ) == c );
    REQUIRE( a.mul( a ) == a );
    REQUIRE( c.div( a ) == c );
  }
    \end{lstlisting}
  \end{block}
  \visible<2>{ \tikz[overlay] \draw[Base08,line width=2pt] (0.7,1.2) rectangle (2.3,3.3); }
\end{frame}

\begin{frame}[fragile]
  \frametitle{Successful Results}
  \begin{block}{Minimal output}
    All tests passed (5 assertions in 1 test case)
  \end{block}
  \begin{block}{Verbose output}
    \begin{verbatim}
  complex.cc:55: 
  PASSED:
    REQUIRE( c.div( a ) == c )
  with expansion:
    1 + 1i == 1 + 1i
    \end{verbatim}
  \end{block}
  \visible<2>{
  \begin{tikzpicture}[overlay]
    \draw[Base08,line width=2pt] (2.5,1.7) ellipse (2 and 0.5);
  \end{tikzpicture}
  }
\end{frame}

\begin{frame}
  \frametitle{Test Assertions}
  From {\color{Base0D} CxxTest}:
  \begin{columns}[onlytextwidth]
    \begin{column}{0.4\textwidth}
      \begin{itemize}
      \item {\ttfamily\small TS\_ASSERT}
      \item {\ttfamily\small TS\_ASSERT\_EQUALS}
      \item {\ttfamily\small TS\_ASSERT\_DIFFERS}
      \item {\ttfamily\small TS\_ASSERT\_DELTA}
      \end{itemize}
    \end{column}
    \begin{column}{0.6\textwidth}
      \begin{itemize}
      \item {\ttfamily\small TS\_ASSERT\_LESS\_THAN}
      \item {\ttfamily\small TS\_ASSERT\_LESS\_THAN\_EQUALS}
      \item {\ttfamily\small TS\_ASSERT\_GREATER\_EQUALS}
      \item {\ttfamily\small TS\_ASSERT\_GREATER\_THAN\_EQUALS}
      \end{itemize}
    \end{column}
  \end{columns}
  \vspace{0.5cm}
  From {\color{Base0D} Catch}:
  \begin{itemize}
  \item {\ttfamily\small REQUIRE}
  \end{itemize}
\end{frame}

\begin{frame}[fragile]
  \frametitle{Failed Results}
  \begin{verbatim}
-----------------------------------------------
Complex numbers
-----------------------------------------------
complex.cc:53
...............................................

complex.cc:60: FAILED:
  REQUIRE( c.div( a ) != c )
with expansion:
  1 + 1i != 1 + 1i

===============================================
1 test case - failed (5 assertions - 1 failed)
  \end{verbatim}
\end{frame}

\begin{frame}[fragile]
  \frametitle{Reporters}
  \footnotesize
  \begin{verbatim}
<Catch name="complex">
  <Group>
    <TestCase name="Complex numbers">
      <Expression success="false" filename="complex.cc" line="60">
        <Original>
          c.div( a ) != c
        </Original>
        <Expanded>
          1 + 1i != 1 + 1i
        </Expanded>
      </Expression>
      <OverallResult success="false"/>
    </TestCase>
    <OverallResults successes="4" failures="1"/>
  </Group>
  <OverallResults successes="4" failures="1"/>
</Catch>
  \end{verbatim}
\end{frame}

\begin{frame}[fragile]
  \frametitle{Unit Testing with ``Catch''}
  Things we would like from unit testing:
  \begin{itemize}
  \vspace{0.5cm}
  \item Easy to implement.
  \vspace{0.5cm}
  \item Descriptive failures.
  \vspace{0.5cm}
  \item Minimum boilerplate.
  \vspace{0.5cm}
  \item Flexibility.
  \end{itemize}
  \begin{tikzpicture}[overlay]
    \node[Base0B,scale=2] at (5.25,4) {\checkmark};
    \node[Base0B,scale=2] at (5.25,2.8) {\checkmark};
    \node[Base0B,scale=2] at (5.25,1.8) {\checkmark};
    \node[Base0B,scale=2] at (5.25,0.6) {\checkmark};
  \end{tikzpicture}
\end{frame}

\begin{frame}[fragile]
  \frametitle{Fixtures}
  \begin{block}{Parse input file}
    \begin{lstlisting}[style=C]
  struct info_t {
    int param_a;
    int param_b;
    ...
  }

  void parse( char const* fn,
              struct info_t* info )
  {
    FILE* f = fopen( fn, "r" );
    if( !f )
      throw std::exception;
    ...
  }
    \end{lstlisting}
  \end{block}
\end{frame}

\begin{frame}[fragile]
  \frametitle{Fixtures}
  \begin{block}{Possible unit tests}
    \begin{lstlisting}[style=C]
  TEST_CASE( "parse/missing file" ) {
  }

  TEST_CASE( "parse/empty file" ) {
  }

  TEST_CASE( "parse/param_a" ) {
  }

  TEST_CASE( "parse/param_b" ) {
  }
    \end{lstlisting}
  \end{block}
\end{frame}

\begin{frame}[fragile]
  \frametitle{Fixtures}
  \begin{block}{Test param\_a}
    \begin{lstlisting}[style=C]
  TEST_CASE( "parse/param_a" ) {
    FILE* f = fopen( "test_file.txt", "w" );
    // Write a set of test data.
    fclose( f );

    struct info_t info;
    parse( "test_file.txt", &info );
    REQUIRE( info.param_a == 30 );
  }
    \end{lstlisting}
  \end{block}
\end{frame}

\begin{frame}[fragile]
  \frametitle{Fixtures}
  \begin{block}{Test param\_b}
    \begin{lstlisting}[style=C]
  TEST_CASE( "parse/param_b" ) {
    FILE* f = fopen( "test_file.txt", "w" );
    // Write a set of test data.
    fclose( f );

    struct info_t info;
    parse( "test_file.txt", &info );
    REQUIRE( info.param_b == 60 );
  }
    \end{lstlisting}
  \end{block}
  \visible<2->{
    \tikz[overlay] \draw[Base08,line width=1pt] (4.5,3.5) ellipse(5 and 0.8);
  }
\end{frame}

\begin{frame}[fragile]
  \frametitle{Fixtures}
  \begin{block}{Input file fixture}
    \begin{lstlisting}[style=C]
  struct parse_fixture {
    std::string   fn;
    struct info_t info;

    parse_fixture() {
      // Should use mkftemp
      fn = "test_file.txt";
      FILE* f = fopen( fn, "w" );
      // Write test contents.
      fclose( f );
    }
  };
    \end{lstlisting}
  \end{block}
\end{frame}

\begin{frame}[fragile]
  \frametitle{Fixtures}
  \begin{block}{Input file fixture}
    \begin{lstlisting}[style=C]
  TEST_CASE_METHOD( parse_fixture, "parse/param_a" ) {
    parse( fn, &info );
    REQUIRE( info.param_a == 30 );
  }

  TEST_CASE_METHOD( parse_fixture, "parse/param_b" ) {
    parse( fn, &info );
    REQUIRE( info.param_b == 60 );
  }
    \end{lstlisting}
  \end{block}
\end{frame}

\begin{frame}
  \frametitle{Python Unit Testing}
  \begin{tabular}{p{3cm}|p{7cm}}
    {\color{Base09}unittest} & Ships with Python, fully featured.\newline {\color{Base0D}\url{http://docs.python.org/2/library/unittest.html}} \vspace{0.1cm} \\
    \hline
    {\color{Base09}doctest} & Interesting take on unit testing, keeps tests with code.\newline {\color{Base0D}\url{http://docs.python.org/2/library/unittest.html}} \vspace{0.1cm} \\
    \hline
    \only<1>{
      {\color{Base09}nose} & Based on unittest but eliminates boilerplate.\newline {\color{Base0D}\url{http://nose.readthedocs.org/en/latest}}
    }
    \only<2>{
      {\color{Base09}nose}\hspace{1cm}{\color{Base0B}$\checkmark$} & Based on unittest but eliminates boilerplate.\newline {\color{Base0D}\url{http://nose.readthedocs.org/en/latest}}
    }
  \end{tabular}
\end{frame}

\begin{frame}[fragile]
  \frametitle{Python Unit Testing}
  \begin{example}
    \begin{lstlisting}[style=Py]
  def setup_func():
    # Setup test fixtures.

  def teardown_func():
    # Teardown test fixtures.

  @with_setup(setup_func, teardown_func)
  def test_something():
    assert my_func() == true

  def test_something_else():
    assert my_other_func() == true
    \end{lstlisting}
  \end{example}
\end{frame}

\begin{frame}[fragile]
  \frametitle{Python Unit Testing}
  \begin{block}{Running nose (autodiscovery)}
    \vspace{0.1cm}
    \hspace{0.2cm}{\ttfamily nosetests}
    \vspace{0.2cm}
  \end{block}
  \begin{block}{Successful output}
    \begin{verbatim}
  ....
  ----------------------
  Ran 4 tests in 0.004s

  OK
    \end{verbatim}
  \end{block}
\end{frame}

\begin{frame}[fragile]
  \frametitle{Python Unit Testing}
  \begin{block}{Failed output}
    \begin{verbatim}
F...
======================================
FAIL: nose_tests.test_something
--------------------------------------
Traceback (most recent call last):
    assert my_func() == False
AssertionError
--------------------------------------
Ran 4 tests in 0.018s
FAILED (failures=1)
    \end{verbatim}
  \end{block}
\end{frame}

\begin{frame}
  \frametitle{Integration Testing}
  \begin{tikzpicture}[overlay]

    \node[bkg,anchor=north west,text width=3.2cm,text depth=6cm,label=above:{\color{Base04}Static}] at (-0.5,2.3) {};
    \node[bkg,anchor=north west,text width=6.8cm,text depth=6cm,label=above:{\color{Base04}Dynamic}] at (3.8,2.3) {};

    \node[nrm2,short,anchor=north west] at (-0.2,1.9) {Reviews};
    \node[nrm2,short,anchor=north west] at (-0.2,0.7) {Walkthroughs};
    \node[nrm2,short,anchor=north west] at (-0.2,-0.5) {Inspections};
    \node[nrm2,short,anchor=north west] at (-0.2,-1.7) {IDE (syntax)};
    \node[nrm2,long,anchor=north west] (F) at (4.1,1.9) {Functional};
    \node[nrm2,long,anchor=north west] (N) at (4.1,-1.6) {Non-functional};

    \coordinate (o) at (6,0.7);
    \coordinate (p) at (6,-2.8);
    \coordinate (g) at (7.52,-3);

    \node[nrm2,tiny,anchor=north west] (a) at (4.1,0.5)  {Unit};
    \node[lit,tiny,anchor=north west] (b) at (5.1,-0.4) {Integration};
    \node[nrm2,tiny,anchor=north west] (c) at (6.72,0.5)  {System};
    \node[nrm2,tiny,anchor=north west] (d) at (7.8,-0.4) {Regression};
    \node[nrm2,tiny,anchor=north west] (e) at (9.4,0.5)  {Sanity};

    \node[nrm2,short,anchor=north west] at (-0.2,-2.9) {Analysis};

    \node[nrm2,tiny,anchor=north west] (f) at (4.1,-3)  {Performance};
    \node[nrm2,tiny,anchor=north west]     at (6.95,-3) {Usability};
    \node[nrm2,tiny,anchor=north west] (h) at (9.18,-3)  {Security};

    \path[draw,color=Base06,-latex] (F) |- (o) -| (a);
    \path[draw,color=Base06,-latex] (F) |- (o) -| (b);
    \path[draw,color=Base06,-latex] (F) |- (o) -| (c);
    \path[draw,color=Base06,-latex] (F) |- (o) -| (d);
    \path[draw,color=Base06,-latex] (F) |- (o) -| (e);

    \path[draw,color=Base06,-latex] (N) |- (p) -| (f);
    \path[draw,color=Base06,-latex] (N) |- (p) -| (g);
    \path[draw,color=Base06,-latex] (N) |- (p) -| (h);

  \end{tikzpicture}
\end{frame}

\begin{frame}
  \frametitle{Integration Testing}
  \vspace{4cm}
  {\Large\color{Base09}Do separate classes/functions interact as they should?}
  \begin{tikzpicture}[overlay,shift={(-2,3)}]
    \node[class] (A) at (3,2)   {\textbf{Class A}\nodepart{second}method\_1};
    \node[class] (B) at (7,2)   {\textbf{Class B}\nodepart{second}method\_2};
    \node[class] (C) at (0.5,0) {\textbf{Class C}\nodepart{second}method\_3};
    \coordinate  (O) at (4.5,-0.5);
    \path[draw,Base06,<-,>=open diamond,thick] (C) -- (O);
    \path[draw,Base06,->,>=latex,thick] (O) -- (A);
    \path[draw,Base06,->,>=latex,thick] (O) -| (B);
    \visible<2->{
      \draw[Base0B,anchor=north west,line width=2pt] (2,-0.5) ellipse(1.8 and 0.8);
      \node[Base08,anchor=north west,scale=8] at (2.25,3.55) {$\times$};
      \node[Base08,anchor=north west,scale=8] at (6.2,3.55) {$\times$};
    }
  \end{tikzpicture}
\end{frame}

\begin{frame}[fragile]
  \frametitle{Integration Testing}
  \begin{block}{Combine complex and newton}
    \begin{lstlisting}[style=C]
  // Complex number implementation.
  struct complex;

  // Perform Newton-Raphson solve.
  // Solution is bounded by [x1,x2].
  // Initial value is x.
  // Solve tolerance is tol, maximum iterations are
  // max_its.
  template< class FuncT, class T >
  T newton( FuncT func, T x1, T x2, T x,
            T tol, int max_its );
    \end{lstlisting}
  \end{block}
\end{frame}

\begin{frame}[fragile]
  \frametitle{Integration Testing}
  \begin{example}
    \begin{lstlisting}[style=C]
  // Example function used with the Newton
  // -Raphson solver.
  //
  //   x^2 - 4
  //
  template< class T >
  struct function
    : public std::unary_function<T>
  {
    T operator()( T x ) const {
      return x*x - T( 4 );
    }
  };
    \end{lstlisting}
  \end{example}
\end{frame}

\begin{frame}[fragile]
  \frametitle{Integration Testing}
  \begin{example}
    \begin{lstlisting}[style=C]
  TEST_CASE( "newton/complex" ) {
    complex x1( 1.0 ), x2( 3.0 );
    function f;
    REQUIRE( newton( f, x1, x2, 1 ) == 2 );
    REQUIRE( newton( f, x1, x2, 3 ) == 2 );
    REQUIRE( newton( f, x1, x2, 2 ) == 2 );
  }
    \end{lstlisting}
  \end{example}
\end{frame}

\begin{frame}
  \frametitle{System Testing}
  \begin{tikzpicture}[overlay]

    \node[bkg,anchor=north west,text width=3.2cm,text depth=6cm,label=above:{\color{Base04}Static}] at (-0.5,2.3) {};
    \node[bkg,anchor=north west,text width=6.8cm,text depth=6cm,label=above:{\color{Base04}Dynamic}] at (3.8,2.3) {};

    \node[nrm2,short,anchor=north west] at (-0.2,1.9) {Reviews};
    \node[nrm2,short,anchor=north west] at (-0.2,0.7) {Walkthroughs};
    \node[nrm2,short,anchor=north west] at (-0.2,-0.5) {Inspections};
    \node[nrm2,short,anchor=north west] at (-0.2,-1.7) {IDE (syntax)};
    \node[nrm2,long,anchor=north west] (F) at (4.1,1.9) {Functional};
    \node[nrm2,long,anchor=north west] (N) at (4.1,-1.6) {Non-functional};

    \coordinate (o) at (6,0.7);
    \coordinate (p) at (6,-2.8);
    \coordinate (g) at (7.52,-3);

    \node[nrm2,tiny,anchor=north west] (a) at (4.1,0.5)  {Unit};
    \node[nrm2,tiny,anchor=north west] (b) at (5.1,-0.4) {Integration};
    \node[lit,tiny,anchor=north west] (c) at (6.72,0.5)  {System};
    \node[nrm2,tiny,anchor=north west] (d) at (7.8,-0.4) {Regression};
    \node[nrm2,tiny,anchor=north west] (e) at (9.4,0.5)  {Sanity};

    \node[nrm2,short,anchor=north west] at (-0.2,-2.9) {Analysis};

    \node[nrm2,tiny,anchor=north west] (f) at (4.1,-3)  {Performance};
    \node[nrm2,tiny,anchor=north west]     at (6.95,-3) {Usability};
    \node[nrm2,tiny,anchor=north west] (h) at (9.18,-3)  {Security};

    \path[draw,color=Base06,-latex] (F) |- (o) -| (a);
    \path[draw,color=Base06,-latex] (F) |- (o) -| (b);
    \path[draw,color=Base06,-latex] (F) |- (o) -| (c);
    \path[draw,color=Base06,-latex] (F) |- (o) -| (d);
    \path[draw,color=Base06,-latex] (F) |- (o) -| (e);

    \path[draw,color=Base06,-latex] (N) |- (p) -| (f);
    \path[draw,color=Base06,-latex] (N) |- (p) -| (g);
    \path[draw,color=Base06,-latex] (N) |- (p) -| (h);

  \end{tikzpicture}
\end{frame}

\begin{frame}
  \frametitle{System Testing}
  \begin{itemize}
  \item Test finished system.
  \vspace{0.3cm}
  \item No different in implementation to integration testing.
  \vspace{0.3cm}
  \item Likely use fixtures to prepare input environment.
  \vspace{0.3cm}
  \item May need complex correctness measures.
    \begin{itemize}
    \item Linear regression.
    \item Approximate comparisons.
    \end{itemize}
  \end{itemize}
\end{frame}

\begin{frame}
  \frametitle{Other Tests}
  \begin{description}
  \item[Regression] Be sure to add new failures to the test suite.
  \item[Smoke] Similar to sanity testing, try out major components to be sure nothing critical is wrong.
  \item[Destructive] Actively attempt to break components.
  \item[Installation] Test installation procedure and correctness on various system configurations.
  \end{description}
\end{frame}
